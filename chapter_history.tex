% --------------------------------------------------------------------
% Cheatsheet:
% \verb!lazy val!   -- esim. lyhyisiin koodinpätkiin
% \begin{verbatim}  -- esim. pitkiin koodinpätkiin
% `` tekstiä ''     -- lainausmerkit
% \textbf{huom!}    -- boldaus
% \begin{quotation}
% \noindent \it     -- quoten lisäys
% \ldots            -- kolme pistettä
% footnote{juh}     -- alaviite
% \citet, \citep
% \citet[s.234]{}   -- viitteet, http://merkel.zoneo.net/Latex/natbib.php
% \begin{sloppypar} -- ahdas teksti
% \clearpage        -- onkkelmien korjaamiseen lukujen kanssa
% --                -- yhdysviiva
% \enumerate
% \itemize          -- listat
% \begin[htb]{figure}
% \begin[htb]{table} - Kuva ja taulukko, (h)ere, (t)op, (b)ottom
% \begin{equation}  -- kaava
% \label{eq:kaava1} -- laabeli
%!TEX root = main.tex

\section{Varhainen historia ja keskeisimmät konseptit}\label{historia}

Laiska evaluointi sai alkusysäyksensä 1970-luvulla. Sarja julkaisuja loi pohjaa ajatukselle laiskoista funktionaalisista kielistä työkaluna käytännönläheiseen ohjelmistokehitykseen. Ajatus esiteltiin ensimmäisenä matemaattisesti lamdakalkyylin, funktionaalisen ohjelmoinnin kannalta keskeisen matemaattisen teorian, näkökulmasta \citep{wadsworth1971semantics}. Viisi vuotta myöhemmin julkaistiin toisistaan riippumatta kolme artikkelia \citep{henderson1976lazy,friedman1976cuns,saslmanualturner}, joissa esiteltiin laiskaa evaluointia ohjelmoinnin perspektiivistä.

1980-luvulla vaihteessa oli uraauurtavaa kehitystä kohti ensimmäisiä laiskoja ohjelmointikieliä. Saman aikaan kehitettiin myös täysin uudenlaisia tietokoneita, jotka kilpailivat alan standardin, Von Neumannin arkkitehtuurin, kanssa. Kuitenkin pidemmän päälle osoittautui, ettei tarvetta erikoisille arkkitehtuureille ole, vaan hyvin laadituilla ohjelmointikielten kääntäjillä voidaan päästä hyviin lopputuloksiin myös Von Neuman -arkkitehtuuria hyödyntävissä tietokoneissa. \citep{hudak2007history}

\subsection{Ei-tiukka semantiikka ja graafireduktio}

Yhdistävää 1980-luvun vaihteessa kehitetyille laiskoille ohjelmointikielille oli, että niissä alettiin hyödyntämään sittemmin vakiintuneita \textit{ei-tiukan semantiikan} ja \textit{graafireduktion} periaatteita. Jos ohjelmointikieli perustuu ei-tiukalle semantiikalle, niin lausekkeella voi olla arvo, vaikka jollakin sen alilausekkeista (pienemmistä lausekkeista, joista lauseke koostuu) ei olisi arvoa. Vastaavasti \textit{tiukkaan semantiikkaan} perustuvat ohjelmointikielet toimivat siten, että jos joltakin alilausekkeelta puuttuu arvo, niin lausekkeellakaan ei ole arvoa, ja tällöin lausekkeen evaluoinnin voi ajatella epäonnistuneen \citep[s. 523]{scott2009programming}.

Tiukkaa semantiikkaa käytetään yleisesti imperatiivisissa ohjelmointikielissä. Useimmissa suosituissa imperatiivisissa kielissä, esimerkiksi Pythonissa, tiukan semantiikan toteutus perustuu funktioiden evaluointijärjestykseen liittyviin sääntöihin. Evaluointi etenee ``sisimmät ensin'' -periaatteella. Siinä funktion kaikki argumentit evaluoidaan ennen kuin funktiota kutsutaan, ja evaluointi etenee sisimmästä funktiokutsusta ulompia funktiokutsuja kohti.

Vastaavasti ei-tiukan semantiikan tyypillisin toteutus, jota tapaa esimerkiksi Haskellissa, perustuu ``uloimmat ensin'' -periaatteeseen. Siinä funktioiden evaluointi etenee uloimmasta funktiokutsusta kohti sisempiä, ja ainoastaan ne funktion argumentit, joita funktio todellisuudessa käyttää, evaluoidaan. Siten Haskellissa on mahdollista tehdä jopa sellaisia funktioita, joka ei tarvitse argumentteja lainkaan, ja siten palauttaa arvon riippumatta siitä, onko parametrilausekkeilla arvoa vai ei. \citep{haskellwikinonstrict}.

Taulukko \ref{table:python_haskell_semantics} demonstroi semantiikan vaikutusta funktioiden evaluointiin. Siinä on toiminallisuudeltaan vastaava koodi suoritetaan sekä Pythonilla että Haskellilla. Funktio \verb!noreturn! aiheuttaa ikuisen silmukan, minkä tähden se ei koskaan palauta arvoa. Haskellilla lausekkeen evaluointi palauttaa arvon, koska \verb!noreturn! -funktiota ei koskaan kutsuta.

\makeatletter
\preto{\@verbatim}{\topsep=0pt \partopsep=0pt }
\makeatother
\newpage
\begin{table}[th]
  \begin{center}
    \begin{tabular}{|p{0.5\textwidth}|p{0.40\textwidth}|}
      \hline
       Python, tiukka semantiikka& Haskell, ei-tiukka semantiikka \\
      \hline
      \footnotesize

      \begin{alltt}
def noreturn(x):
    while True:
        x = -x
    return x # not reached

def even(x):
  return x % 2 == 0

> any(even(n) for n in [3, 2, noreturn(6)])
\(\Rightarrow\) (ei palauta arvoa)
    \end{alltt}
      &\footnotesize\begin{alltt}
noreturn :: Integer -> Integer
noreturn x = negate (noreturn x)

> any . even . [3, 2, noreturn 6]
\(\Rightarrow\) True

\end{alltt}

      \textit{Katsotaan, haluanko tässä vielä käydä evaluoinnin välivaiheita läpi, vai viittaanko samaan esimerkkiin myöhemmin uudelleen.}\\
      \hline
    \end{tabular}
    \caption{Esimerkki kielen semantiikan vaikutuksesta evaluointijärjestykseen }
    \label{table:python_haskell_semantics}
  \end{center}
\end{table}

Graafireduktio on tapa ei-tiukan semantiikan toteuttamiseen. Se esittää lausekkeet verkon muodossa, mikä mahdollistaa toistuvien lausekkeiden jakamisen muiden lausekkeiden kesken \citep{hudak1989conception}. Esimerkiksi lausekkeen \verb!(1+2)*(1+2)! verkkoesityksessä lauseke \verb!(1+2)! pystytään jakamaan, minkä myötä sen arvo tarvitsee evaluoida vain kerran. % LÄHDE?

Ei-tiukan semantiikan ``uloimmat ensin'' -suoritusjärjestys ja graafireduktiossa tapahtuva lausekkeiden jakaminen luovat perustan laiskalle evaluoinnille. Lausekkeen arvoa ei lasketa ennen kuin on tarve, eikä sitä turhaan lasketa uudelleen, jos sitä käytetään myöhemmin uudelleen.

\subsection{Haskellin ja puhtaan funktionaalisen ohjelmoinnin kehitys}

\citet{hudak2007history} kuvaa, kuinka 1980-luvun puolivälissä laiskaa evaluointia hyödyntäviä ohjelmointikieliä alkoi olla ruuhkaksi asti. Useimmat kielistä soveltuivat vain kapeaan määrään käyttökohteita, ja niillä ei ollut riittävää määrää käyttäjiä suuren suosion saavuttamiseksi. Kuitenkin artikkelin kirjoittajat olivat tällöin sitä mieltä, että kielet muistuttivat ominaisuuksiltaan hyvin paljon toisiaan. Alkoi kehittyä ajatus siitä, että olisi hyvä luoda yksi, yleinen kieli, joka korvaisi kerralla monia aikaisempia kieliä.

Tämä johti Haskell-ohjelmointikielen kehityksen aloittamiseen. Siitä vastasi Haskell-komitea, jossa vaikutti monia aikaisempien laiskaa evaluointia hyödyntäneiden kielien suunnittelijoita. Komitea onnistui keräämään yhteen aikaisemmin erillään samaa aihepiiriä tutkineita, mikä kiihdytti laiskan evaluoinnin parissa tapahtunutta tieteellistä kehitystä. Alkuvaiheen kehitystyö oli onnistunutta, ja Haskellista kehittyi etenkin tietojenkäsittelytieteen akateemisen tutkimuksen parissa suosittu kieli.

Haskell ei määritelmällisesti ole laiska ohjelmointikieli, vaan jo Haskell 1.1 -raportissa \citep{yale1991report} se määriteltiin ainoastaan ei-tiukkaa semantiikkaa käyttäväksi kieleksi. Käytännössäkään kieli ei ole täysin laiska, vaan Haskellin kääntäjät tekevät useita nopeusoptimointeja suorittamalla tiettyjä osia koodista tiukkaan semantiikan säännöllä. Useimmat kielen rakenteet noudattavat kuitenkin oletusarvoisesti laiskan evaluoinnin periaatteita.

Laiskassa evaluoinnissa koodia ei suoriteta imperatiivisten kielien tapaan rivi riviltä, vaan suoritusjärjestys määräytyy tarpeen mukaan. IO-operaatioita, eli esimerkiksi näytölle tulostamista tai käyttäjän syötteen odottamista, ei pystytä suorittamaan halutussa järjestyksessä pelkkien funktiokutsujen avulla. Tästä seurasi, että Haskellista tuli \textit{puhtaasti funktionaalinen} ohjelmointikieli. Käsitteellä tarkoitetaan, että kielen funktiokonstruktiot ovat funktioita matemaattisessa mielessä: funktion kutsuminen samoilla argumenteilla palauttaa aina saman arvon, eikä funktioilla voi suorittaa IO-operaatioita tai muita ohjelman tilaa muuttavia operaatioita.

Laiskan evaluoinnin periaatteiden noudattaminen on johtanut moniin innovaatioihin funktionaalisen ohjelmoinnin saralla. Koska Haskell on puhtaasti funktionaalinen kieli, täytyi kehittää kokonaan uusia konsepteja, jotta esimerkiksi IO-operaatioiden suorittaminen halutussa järjestyksessä ja niiden ketjuttaminen toisiinsa onnistuu. Tunnetuimmaksi konseptiksi on noussut \textit{monadi}. Alkuperäisessä monadin konseptin esitelleessä artikkelissa sitä kutsutaan ``imperatiivisen funktionaalisen ohjelmoinnin'' työkaluksi, mikä kuvaa monadien mahdollistamia sekventiaalisia IO-operaatioita \citep{PeytonJones199371}.

Niin monadi kuin monet muutkin Haskelliin kehitetyt puhtaan funktionaalisen ohjelmoinnin konseptit ovat levinneet myös sellaisiin ohjelmointikieliin, jotka ovat evaluointistrategialtaan tiukkoja. Siten laiskan evaluoinnin vaalimisen Haskellissa voi ajatella olleen hyödyksi tietojenkäsittelytieteen kehitykselle laajemminkin.

\textit{Mainintaa tähän osioon vielä (1) thunkeista eli evaluoimattomista arvoista liittyen Haskellin evaluointiin ja (2) pysyvätilaisista muuttujista liittyen Haskellin puhtaasti funktionaaliseen luonteeseen? }

\subsection{Päättymättömät tietorakenteet ja kontrollirakenteet}

Jo 1980-luvun varhaisessa tutkimuksessa käsiteltiin laiskan evaluoinnin mahdollistamia laiskoja tietorakenteita, etenkin ``päättymättömiä tietorakenteita''. Näitä ovat muun muassa päättymättömät listat, puurakenteet ja tapahtumavirrat. Ne ovat tyypillisesti rekursion avulla toteutettuja, ja laiskan evaluoinnin ansiosta tietorakenteessa tarvitsee mennä vain niin ``syvälle'' kuin on tarvetta.

\newpage

Eräs klassinen esimerkki on matematiikan äärettömiä lukujonojen kuvaaminen. Seuraavassa on Fibonaccin lukujono ilmaistuna rekursiivisesti Haskellilla. Huomionarvoista on, kuinka paljon koodi muistuttaa tapaa, jolla rekursiivinen lukujono ilmaistaisiin matematiikassa:

\begin{alltt}
fib 0 = 0
fib 1 = 1
fib n = fib (n-1) + fib (n-2)
\end{alltt}

Laiska evaluointi mahdollistaa myös omien kontrollirakenteiden kirjoittamisen. Esimerkiksi if-ehtolauseesta on Haskellissa helppoa tehdä oma versio, jossa ehdon täyttymisen perusteella evaluoidaan vain jompikumpi vaihtoehtoisista lausekkeista:

\begin{alltt}
% Käyttö: myIf condition onTrue onFalse
myIf :: Bool -> a -> a -> a
myIf True  x _ = x
myIf False _ y = y
\end{alltt}

% Tässä olisi ehkä kiva mainita toki siitä, että täsmäkielen rakentamiseen vaikuttaa toki monta muutakin tekijää.
Omien kontrollirakenteiden tuki helpottaa kieleen upotettujen \textit{täsmäkielten} rakentamista. Täsmäkielillä tarkoitetaan pieniä, tavanomaisesti deklaratiivisia kieliä, jotka ovat hyvin ilmaisuvoimaisia tietyssä ongelma-avaruudessa \citep{van2000domain}. Täsmäkielet toteutetaan usein ohjelmointikielen apukirjastoina, ja täsmäkieli on siinä tapauksessa yhdistelmä ohjelmointikielen valmiita ominaisuuksia ja apukirjastossa siihen laadittuja lisäyksiä.

Seuraavassa luvussa siirrytään tarkastelemaan sitä, millaisia sovellutuksia laiskalle evaluoinnille on löydetty erilaisissa ohjelmointikielissä ja ohjelmointiin liittyvissä \mbox{konteksteissa.}
