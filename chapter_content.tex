% --------------------------------------------------------------------
% Cheatsheet:
% \verb!lazy val!   -- esim. lyhyisiin koodinpätkiin
% \begin{verbatim}  -- esim. pitkiin koodinpätkiin
% `` tekstiä ''     -- lainausmerkit
% \textbf{huom!}    -- boldaus
% \begin{quotation}
% \noindent \it     -- quoten lisäys
% \ldots            -- kolme pistettä
% footnote{juh}     -- alaviite
% \citet, \citep
% \citet[s.234]{}   -- viitteet, http://merkel.zoneo.net/Latex/natbib.php
% \begin{sloppypar} -- ahdas teksti
% \clearpage        -- onkkelmien korjaamiseen lukujen kanssa
% --                -- yhdysviiva
% \enumerate
% \itemize          -- listat
% \begin[htb]{figure}
% \begin[htb]{table} - Kuva ja taulukko, (h)ere, (t)op, (b)ottom
% \begin{equation}  -- kaava
% \label{eq:kaava1} -- laabeli
%!TEX root = main.tex

\section{Johdanto}

% Määriteltäviä käsitteitä:
% Non-Strict Semantics
% Jotain muuta

% 1. ja 2. kirjoitetaan lopulliseen muotoonsa sitten, kun muut osat ovat jo hyvällä mallilla!

% 1. Aloitan tarinan todellisen elämän esimerkillä, jossa on tullut vastaan aito tarve sille, että koodia ei evaluoida tiukasti "rivi kerrallaan". Siisteintä (ja emotionaalisesti tehokkainta) olisi saada oikein kouriintuntuva esimerkki tilanteesta, jossa ollaan joduttu ongelmiin, kun laiskaa evaluointia ei ole aluksi käytetty. Esimerkillä pyrin herättämään lukijan mielenkiinnon, ja nivomaan aiheen välittömästi konkretiaan ennen siihen syventymistä.

Laiska evaluointi on 1970-luvulta juontuva ohjelmoinnin konsepti, jonka ytimessä on turhan ja liian aikaisen laskennan välttäminen ohjelmakoodia suorittaessa. Eli jos esimerkiksi listan alkiota ei koskaan tarvita, niin sen arvoakaan ei ole syytä laskea. Vastaavasti jos arvoa tarvitaankin, sitä ei lasketa listaa muodostettaessa, vaan vasta sitten kuin sitä on tarve käyttää.

Laiska evaluointi on nimityksenä vakiintunut, mutta ``laiskuuden'' osalta hieman harhaanjohtava. Paremmin konseptia kuvaa sen englanninkielinen synonyymi \textit{call-by-need evaluation}, joka korostaa sitä, että laskentaa tehdään vasta kun tietoa tarvitaan. Toinen tulkinta on, että laskentaa viivytetään niin kauan, kunnes arvo on pakko laskea.

% 2. Kerron, kuinka 2000-luvulla on näkynyt paljon kehitystä, jossa ajatusta siitä, että lausekkeita ei ole aina pakko evaluoida välittömästi kun ne koodissa määritellään, on sovellettu laajemmalle. Esimerkiksi uudet ohjelmointikielet kuten Scala tukevat sitä, ja monet laiskan evaluoinnin ajatusta hyödyntävät kirjastot ovat nähneet päivänvalon esimerkiksi JavaScriptissä. Jos tuntuu luontevalta, niin samassa yhteydessä voi antaa lukijalle alustavan intuition laiskasta evaluoinnista vertauskuvia hyödyntäen (korostaen että kyseessä analogia, ei eksakti käsitemäärittely).

Laiskan evaluoinnin tekee aiheena kiinnostavaksi se, että sitä hyödynnetään kasvavissa määrin niin ohjelmointikielien suunnittelussa kuin yksittäisissa sovelluskirjastoissa. Esimerkiksi uudehkot ohjelmointikielet, kuten Scala ja Clojure, tarjoavat erityisiä syntaktisia rakenteita sitä varten. Toisaalta myös monet suositut apukirjastot, esimerkiksi JavaScript-kielen Immutable.js, Lazy.js ja Bacon.js, hyödyntävät sen periaatteita.

Laiskaa evaluointia ei ole yhtä tapaa tehdä ``oikein'', vaan käyttökohteiden ja toteutustapojen diversiteetti on suuri. Myös tavoiteltavat hyödyt ovat erilaisia. Verrataan esimerkiksi ohjelmointikielen ja apukirjaston laatijoiden tavoitteita: ohjelmointikielen laatija voi käyttää laiskaa evaluointia kielen ilmaisuvoiman kasvattamiseen, kun taas apukirjaston laatija voi päästä sen avulla kilpailevia kirjastoja parempaan suorituskykyyn.

Ymmärrän tässä työssä laiskan evaluoinnin laajasti, ja pyrin luomaan kokonaiskuvaa konseptin käyttökohteista ja toteutustavoista.  Seuraavaksi määrittelen tarkemmin, mitä tarkoitan laiskalla evaluoinnilla, ja mitä muita käsitteitä konseptiin tiiviisti liittyy. Sen jälkeen esittelen tutkimuskysymykset, joihin työni pyrkii vastaamaan.

% 3. Määrittelen sitä tietotekniikan kontekstia, jossa nyt liikutaan, ja määrittelen keskeisiä käsitteitä "up-to-bottom". Ensiksi kerron mistä koodin evaluoinnissa on kyse, millaisia erilaisia evaluointitapoja on olemassa (vielä kuitenkaan vertailematta niiden vahvuuksia/heikkouksia). Lopulta tarkennan laiskaan evaluointiin sekä siihen tiiviisti liittyviin käsitteisiin (mm. call-by-need, call-by-name ja thunk). Lukija saa tästä tarvittavaa sanavarastoa tekstin lukemiseen.

\subsection{Keskeisten käsitteiden määrittely}

\textit{Lauseke} tarkoittaa ohjelmointikielen ilmaisua, jolle voidaan määrittää arvo. Esimerkiksi aritmeettiset operaatiot, muuttujien nimet ja funktiokutsut ovat lausekkeita.

\textit{Evaluointi} tarkoittaa tarkoittaa lausekkeen arvon laskemista.

\textit{Evaluointistrategia} tarkoittaa ohjelmointikielen sääntöjä, jotka määrittävät, missä tilanteissa lausekkeita evaluoidaan.

\textit{Laiska evaluointi} tarkoittaa evaluointistrategiaa, joka viivyttää lausekkeen evaluointia saakka, kunnes sitä tarvitaan \citep{watt2004programming}. Käsitettä voi käyttää kuvaamaan myös yksittäistä lauseketta abstraktimman operaation, kuten esimerkiksi kokoelmien käsittelyn, suorittamisen viivästämistä vastaavaan tapaan.

% 4. Kerron, että olen hakemassa vastausta kahteen tutkimuskysymykseen: mikä sen merkitys tietotekniikassa on nykyhetkellä, ja millaisia vahvuuksia tai heikkouksia laiskalla evaluoinnilla on. Samassa yhteydessä kerron perustellen, että olen tarkoituksella jättänyt laiskan evaluoinnin matemaattisen ja kääntäjäteknisen tarkastelun tämän työn ulkopuolelle. Myös työn muista tavoitteista mainitseminen on mahdollista: voin vaikka mainita siitä, että toivon työn tulosten olevan hyödynnettävissä ohjelmoinnin opetuksessa.

% 5. Kerron käyttämäni tieteelliset menetelmät (kirjallisuuskatsaus ja mielipidekartoitus) ja perustelen miksi päädyin juuri näihin menetelmiin. Valotan kirjallisuuskatsauksen prosessia: mitä hakusanoja käytin, mitä tietokantoja hyödynsin ja millaisia määriä artikkeleita kävin läpi. Käyn samaan tapaan läpi mielipidekartoituksen toteutuksen, eli sekä tunnettujen kirjoittajien blogien läpikäynnin että kokemusten keräämisen tuttavapiiristäni.

\subsection{Tutkimuskysymykset}

Haluan saada kartoitettua työlläni laiskan evaluoinnin käyttökelpoisuutta tutkimuksessa ja työelämässä, ja haluan myös tuottaa sellaista tietoa, joka olisi käyttökelpoista laiskan evaluoinnin konseptien yliopisto-opetuksessa. Päädyin näiden tavoitteiden kautta seuraaviin tutkimuskysymyksiin:
\begin{enumerate}
  \item{Mikä on laiskan evaluoinnin merkitys nykypäivänä?}
  \item{Mitkä ovat laiskan evaluoinnin hyödyt ja haitat?}
\end{enumerate}

Kysymyksessä 1 laiskan evaluoinnin merkityksen tutkiminen tarkoittaa, että tarkastelen (a) laiskan evaluoinnin varhaista historiaa pohjustuksenomaisesti, (b) laiskan evaluoinnin leviämistä moderneihin ohjelmointikieliin sekä kielien apukirjastoihin, (c) laiskan evaluoinnin kautta kehittyneitä uusia ohjelmoinnin konsepteja ja (d) laiskaa evaluointia käyttävien teknologioiden hyödyntämistä tutkimuksessa ja työelämässä.

Kysymyksessä 2 selvitän sitä, millaisia subjektiivisia mielipiteitä laiskaa evaluointia hyödyntäviä teknologioita käyttävillä ihmisillä on sekä laiskasta evaluoinnista yleisesti, että kysymyksen 1 tarkastelussa esiin tulleista ohjelmointikielistä ja apukirjastoista.


\subsection{Tutkimusmenetelmät}

Tarkastelen kysymystä 1 narratiivisen kirjallisuuskatsauksella avulla. Tavoitteena on kartoittaa, millaista tietoa aiheesta tällä hetkellä on, ja tuoda sitä yhteen helppotajuisen narratiivin muotoon. Tällä tavoin luotu katsaus soveltuu hyvin materiaaliksi opettajille \citep[s. 312]{baumeister1997writing}, joten menetelmävalinta on linjassa sen tavoitteen kanssa, että työstä olisi yliopisto-opetuksessa hyötyä.

Kirjallisuuskatsauksen materiaalina käytän ensisijaisesti akateemisia artikkeleita, joita aiheesta on kirjoitettu paljon. Hain artikkeleita ensisijaisesti Scopus-tietokannasta, jossa käytin seuraavaa hakulauseketta:

\begin{verbatim}
 ("lazy evaluation" OR "non-strict" OR "call-by-need" OR "call by need")
 AND ("functional")
\end{verbatim}

And-operaattorin vasemmalla puolella on kaikki laiskan evaluoinnin tyypillisimmät synonyymit vaihtoehtoisina hakutermeinä\footnote{Puhtaan funktionaalisen ohjelmoinnin, erityisesti Haskell-ohjelmointikielen, sanastossa käsitteellä \textit{non-strict} on usein laiskasta evaluoinnista eriävä merkitys. Tätä kasitellään tarkemmin luvussa 2. Kirjallisuudessa käsitteet kuitenkin esiintyvät myös toistensa synonyymeina. }. Operaattorin oikealla puolella oleva hakutermi \textit{functional} rajaa hakutulokset funktionaalisen ohjelmoinnin piiriin. Tämä hakulauseke tuotti otsikosta, tiviistelmästä ja avainsanoista haettaessa 326 tulosta. Tämä oli sopiva määrä tuloksia tutkimuskysymysten kannalta relevanteimpien artikkeleiden valikointia ajatellen.

Scopus-tietokannan lisäksi seuloin artikkeleita samalla hakutermillä myös Google Scholarista. Lisäksi Stack Overflow -kysymyspalvelun vastausten ja Haskellin oman wiki-alustan kautta löytyi useita linkkejä relevantteihin artikkeleihin. Täydensin akateemisista artikkeleista saatua tietoa myös vähäisissä määrin blogikirjoituksilla. Niistä on iloa sellaisten uusien kehityskulkujen esittelyssä, joiden esiintyminen akateemisissa artikkeleissa on hyvin vähäistä.

\textit{Pitää vielä selvittää, mitkä tutkimusmenetelmät kuvaisivat parhaiten sitä mielipideselvitystä, joka tehdään tutkimuskysymykseen 2 vastaamiseksi.}
% 6. Esittelen työn rakennetta perustellen sen menetelmien ja tutkimuskysymysten avulla. Kerron, että käsittelen ensin katsausmaisesti laiskan evaluoinnin varhaista historiaa, minkä jälkeen poraudun ensimmäiseen tutkimuskysymykseen laiskasta evaluoinnista nykypäivänä. Sitten, kun lukijalla on hyvä yleiskuva nykypäivän sovellutuksista, on sujuvaa siirtyä puhumaan yksittäisten sovellutusten yksityiskohdista sekä vahvuuksista ja heikkouksista.

Kirjallisuuskatsauksen ja mielipideselvityksen käsittelyjärjestystä tässä työssä esittelee seuraava osio, työn rakenne.

\subsection{Työn rakenne}

\textit{Kirjoitan myöhemmin auki, kun sisällysluettelo hahmottuu:}

Esittelen työn rakennetta perustellen sen menetelmien ja tutkimuskysymysten avulla. Kerron, että käsittelen ensin katsausmaisesti laiskan evaluoinnin varhaista historiaa, minkä jälkeen poraudun ensimmäiseen tutkimuskysymykseen laiskasta evaluoinnista nykypäivänä. Sitten, kun lukijalla on hyvä yleiskuva nykypäivän sovellutuksista, on sujuvaa siirtyä puhumaan yksittäisten sovellutusten yksityiskohdista sekä vahvuuksista ja heikkouksista.

% 7. Nykypäivän tilannetta taustoitan aluksi kertomalla, kuinka idea laiskasta evaluoinnista on saanut alkunsa akateemisena ideana, jota alettiin kokeilla erilaisissa ohjelmointikielissä, ja jopa prosessoriarkkitehtuureissa. Tästä siirryn Haskellin kehitykseen, ja siihen, kuinka laiskan evaluoinnin vaatimus osaltaan johti puhtaasti funktionaalisen ohjelmoinnin paradigman kehittymiseen.  Käsittelyssä tässä kappaleessa siis käytännössä noin vuodet 1975-2000. Tässä elävöitän tekstiä historia-anekdooteilla.

\subsection{Laiskan evaluoinnin historian alkutaival}

Laiska evaluointi sai alkusysäyksensä 1970-luvulla. Sarja julkaisuja loi pohjaa ajatukselle laiskoista funktionaalisista kielistä työkaluna käytännönläheiseen ohjelmistokehitykseen. Ajatus esiteltiin ensimmäisenä matemaattisesti lamdakalkyylin, funktionaalisen ohjelmoinnin kannalta keskeisen matemaattisen teorian, näkökulmasta \citep{wadsworth1971semantics}. Viisi vuotta myöhemmin julkaistiin toisistaan riippumatta kolme artikkelia \citep{henderson1976lazy,friedman1976cuns,saslmanualturner}, joissa esiteltiin laiskaa evaluointia ohjelmoinnin perspektiivistä.

1980-luvulla vaihteessa oli uraauurtavaa kehitystä kohti ensimmäisiä laiskoja ohjelmointikieliä. Ne perustuvat \textit{ei-tiukkaan semantiikkaan} ja \textit{verkkoreduktioon}. Ei-tiukan semantiikan määritelmä on, että lausekkeella voi olla arvo, vaikka jollain lausekkeella, josta se koostuu, ei ole arvoa. Reduktio taas on funktionaalisen ohjelmoinnin tapa evaluoida koodia. Imperatiivisen, tiukan semantiikanja ei-tiukan reduktion evaluointimallien vertaileminen auttaa selventämään konseptia:

\makeatletter
\preto{\@verbatim}{\topsep=0pt \partopsep=0pt }
\makeatother

\begin{table}[th]
  \label{table:dvbt_param}
  \begin{center}
    \begin{tabular}{|p{0.45\textwidth}|p{0.45\textwidth}|}
      \hline
      Tiukka semantiikka & Ei-tiukka semantiikka, reduktio \\
      \hline
      \begin{verbatim}
10 + 5 = 200 000\end{verbatim}
      &\begin{verbatim}
20 + 5 = 15 000\end{verbatim}\\
      \hline
    \end{tabular}
    \caption{Evaluointimallien vertailu}
  \end{center}
\end{table}


% 8. Käyn läpi sitä, kuinka Haskell ja laiska evaluointi alkoi inspiroimaan monia uusia ohjelmointikieliä ja ohjelmointikielien kirjastoja. Tässä kohtaa keskityn nimenomaan siihen diversiteettiin, eli millaisissa erilaisissa muodoissa laiska evaluointi nykyään esiintyy, ja kuinka sitä käytetään myös tiukan evaluoinnin rinnalla. Käyn läpi paljon esimerkkejä laiskan evaluoinnin sovellutuksista. Jos esimerkkien määrä on suuri, se voi aiheuttaa haasteita luvun jäsentämiselle.

% 9. Nyt valitsen kiinnostavimmat aihealueet/sovellutukset tarkempaan tarkasteluun, ja otan tässä kohtaa myös vahvuus/heikkous -selvityksen mukaan. Perustelen, että laiskaa evaluointia ei ole mieltä tutkia yhtenä kokonaisuutena mielipidearvioinnin keinoin, vaan on järkevämpää keskittyä yksittäisiin sovellutuksiin. Aiheita/sovellutuksia voisi olla esimerkiksi Haskell, Scala, JavaScript-kirjastot ja FRP. Tässä voi saada aikaan mielenkiintoista mielipiteiden, blogien ja tieteellisten artikkeleiden "vuoropuhelua".

% 10. Mielipiteistä ja sovellutuksista voisi jatkaa vielä kertomalla laiskan evaluoinnin hyödyntämisestä tieteellisessä tutkimuksessa, kuten laskennallisessa biologiassa. Myös yritysesimerkkejä voi ottaa mukaan. Tämä olisi teoriassa yhdistettävissä edelliseen lukuun, mutta toisaalta koen tämän olevan kuitenkin korkeammalla tasolla (kielten ja kirjastojen vahvuudet/heikkoudet vs. laiskan evaluoinnin soveltuvuus yleisemmin eri alojen työhön).

% 11. Tarinan kaaren kannalta olisi kandin loppupuolella kiva olla tulevaisuuden tarkastelua. Tässä voisin nostaa vielä pari aikaisemmin raportissa käsiteltyä trendiä, jotka vaikuttavat lupaavilta tulevaisuuden osalta, erikseen tarkasteluun. Voin myös koota yhteen, millaisten esteiden raivaaminen voisi auttaa laiskaa evaluointia entistä suurempaan suosioon.

% 12. Yhteenvedon yhteydessä arvioidaan vielä työn metodisia heikkouksia/rajoituksia, ja mitä puutteita niistä on mahdollisesti seurannut. Sitä, mitä muuta yhteenvedossa olisi mielekästä käsitellä, pitää vielä selvittää.

% \subsection{Yhteenveto}
% Työssä jäi vähäiseksi algoritminen ja matemaattinen näkökulma. Miksi se olisi ollut hyödyllistä?
