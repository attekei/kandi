% ---------------------------------------------------------------------
% -------------- PREAMBLE ---------------------------------------------
% ---------------------------------------------------------------------
\documentclass[12pt,a4paper,finnish,oneside]{article}
%\documentclass[12pt,a4paper,finnish,twoside]{article}
%\documentclass[12pt,a4paper,finnish,oneside,draft]{article} % luonnos, nopeampi

% Valitse 'input encoding':
%\usepackage[latin1]{inputenc} % merkistökoodaus, jos ISO-LATIN-1:tä.
\usepackage[utf8]{inputenc}   % merkistökoodaus, jos käytetään UTF8:a
% Valitse 'output/font encoding':
%\usepackage[T1]{fontenc}      % korjaa ääkkösten tavutusta, bittikarttana
\usepackage{ae,aecompl}       % ed. lis. vektorigrafiikkana bittikartan sijasta
% Kieli- ja tavutuspaketit:
\usepackage[english,swedish,finnish]{babel}
% Kurssin omat asetukset aaltosci_t.sty:
\usepackage{aaltosci_t}
% Jos kirjoitat muulla kuin suomen kielellä valitse:
%\usepackage[finnish]{aaltosci_t}
%\usepackage[swedish]{aaltosci_t}
%\usepackage[english]{aaltosci_t}
% Muita paketteja:
\usepackage{alltt}
\usepackage{amsmath}   % matematiikkaa
\usepackage{calc}      % käytetään laskurien (counter) yhteydessä (tiedot.tex)
\usepackage{eurosym}   % eurosymboli: \euro{}
\usepackage{url}       % \url{...}
\usepackage{listings}  % koodilistausten lisääminen
\usepackage{algorithm} % algoritmien lisääminen kelluvina
\usepackage{algorithmic} % algoritmilistaus
\usepackage{hyphenat}  % tavutuksen viilaamiseen liittyvä (hyphenpenalty,...)
\usepackage{supertabular,array}  % useampisivuinen taulukko
\usepackage{etoolbox}

% Koko dokumentin kattavia asetuksia:

% Tavutettavia sanoja:
%\hyphenation{vää-rin me-ne-vi-en eri-kois-ten sa-no-jen tavu-raja-ehdo-tuk-set}
% Huomaa, että ylläoleva etsii tarkalleen kyseisiä merkkijonoja, eikä
% ymmärrä taivutuksia. Paikallisesti tekstin seassa voi myös ta\-vut\-taa.

% Rangaistaan tavutusta (ei toimi?! Onko hyphenat-paketti asennettu?)
\hyphenpenalty=10000   % rangaistaan tavutuksesta, 10000=ääretön
\tolerance=1000        % siedetään välejä riveillä
% titlesec-paketti auttaa, jos tämän mukana menee sekaisin

% Tekstiviitteiden ulkoasu.
% Pakettiin natbib.sty/aaltosci.bst liittyen katso esim.
% http://merkel.zoneo.net/Latex/natbib.php
% jossa selitykset citep, citet, bibpunct, jne.
% Valitse alla olevista tai muokkaa:
\bibpunct{(}{)}{;}{a}{,}{,}    % a = tekijä-vuosi (author-year)
%\bibpunct{[}{]}{;}{n}{,}{,}    % n = numero [1],[2] (numerical style)

% Rivivälin muuttaminen:
\linespread{1.24}\selectfont               % riviväli 1.5
%\linespread{1.24}\selectfont               % riviväli 1, kun kommentoit pois

% ---------------------------------------------------------------------
% -------------- DOCUMENT ---------------------------------------------
% ---------------------------------------------------------------------

\begin{document}

% -------------- Tähän dokumenttiin liittyviä valintoja  --------------

%\raggedright         % Tasattu vain vasemmalta, ei tavutusta
% ----------------- joitakin makroja ----------------------------------
%
% \newcommand{\sinunKomentosi}[argumenttienMäärä]{komennot%
% voiJakaaRiveille%
% jaArgumenttienViittaus#1,#2,#argumenttienMäärä}

% Joskus voi olla tarpeen kommentoida jotakin. Ei suositella. 
% Äläkä unohda lopulliseen! 
\newcommand{\Kommentti}[1]{\fbox{\textbf{OMA KOMMENTTI:} #1}}
% Käyttö: Kilometri on 1024 metriä. \Kommentti{varmista tämä vielä}.
% Eli newcommand:n komentosanan jälkeen hakasaluissa argumenttien lkm,
%  ja argumentteihin viitataa #1, #2, ...

%  Comment out this \DRAFT macro if this version no longer is one!  XXX
%\newcommand{\DRAFT}{\begin{center} {\it DRAFT! \hfill --- \hfill DRAFT!
%\hfill --- \hfill DRAFT! \hfill --- \hfill DRAFT!}\end{center}}

%  Use this \DRAFT macro in the final version - or comment out the 
%  draft-command
% \newcommand{\DRAFT}{~}

% %%%%%%%% MATEMATIIKKA %%%%%%%%%%%%%%%%%

% Määrätty integraali
\newcommand{\myInt}[4]{%
\int_{#1}^{#2} #3 \, \textrm{d}{#4}}

% http://kapsi.fi/jks/satfaq/
%\newcommand{\vii}{\mathop{\Big/}}
%\newcommand{\viiva}[2]{\vii\limits_{\!\!\!\!{#1}}^{\>\,{#2}}}
%%\[ \intop_0^{10} \frac{x}{x^2+1} \,\mathrm{d}x
%%= \viiva{0}{10} \frac{1}{2}\ln(x^2+1) \]

% matht.sty, Simo K. Kivelä, 01.01.2002, 07.04.2004, 19.11.2004, 21.02.2005
% Kokoelma matemaattisten lausekkeiden kirjoittamista helpottavia
% määrittelyjä.

% 07.04.2004 Muutama lisäys ja muutos tehty: \ii, \ee, \dd, \der,
% \norm, \abs, \tr.
%
% 19.11.2004 Korjattu määrittelyjä: \re, \im, \norm;
% lisätty \trp (transponointi), \hrm (hermitointi), \itgr (rakenteellinen
% integraali), ympäristö Cmatrix (hakasulkumatriisi);
% vanha transponointi \tr on mukana edelleen, mutta ei suositella.

% Pakotettu rivinvaihto, joka voidaan tarvittaessa määritellä
% uudelleen: 

%\newcommand{\nl}{\newline}

% Logiikan symboleja (<=> ja =>) hieman muunnettuina:

%\newcommand{\ifftmp}{\;\Leftrightarrow\;}
%\newcommand{\impltmp}{\DOTSB\;\Rightarrow\;}

% 'siten, että' -lyhenne ja hattupääyhtäläisyysmerkki vastaavuuden
% osoittamiseen: 

%\newcommand{\se}{\quad \text{siten, että} \quad}
%\newcommand{\vs}{\ {\widehat =}\ }

% Lukujoukkosymbolit:

%\newcommand{\N}{\ensuremath{\mathbb N}}
%\newcommand{\Z}{\ensuremath{\mathbb Z}}
%\newcommand{\Q}{\ensuremath{\mathbb Q}}
%\newcommand{\R}{\ensuremath{\mathbb R}}
%\newcommand{\C}{\ensuremath{\mathbb C}}

% Reaali- ja imaginaariosa, imaginaariyksikkö:

%\newcommand{\re}{\operatorname{Re}}
%\newcommand{\im}{\operatorname{Im}}
%\newcommand{\ii}{\mathrm{i}}

% Differentiaalin d, Neperin luku:

%\newcommand{\dd}{\mathrm{d}}
%\newcommand{\ee}{\mathrm{e}}

% Vektorimerkintä, joka voidaan tarvittaessa määritellä uudelleen
% (tämä tekee vektorit lihavoituina):

%\newcommand{\V}[1]{{\mathbf #1}}

% Kulmasymboli:

%\renewcommand{\angle}{\sphericalangle}

% Vektorimerkintä, jossa päälle pannaan iso nuoli;
% esimerkiksi \overrightarrow{AB} (tämmöisiä olemassaolevien
% symbolien uudelleenmäärittelyjä ei kyllä pitäisi tehdä):

%\renewcommand{\vec}[1]{\overrightarrow{#1}}

% Vektoreiden vastakkaissuuntaisuus:

%\newcommand{\updownarrows}{\uparrow\negthinspace\downarrow}

% Itseisarvot ja normi:

%\newcommand{\abs}[1]{{\left\vert#1\right\vert}}
%\newcommand{\norm}[1]{{\left\Vert #1 \right\Vert}}

% Transponointi ja hermitointi:

%\newcommand{\trp}[1]{{#1}\sp{\operatorname{T}}}
%\newcommand{\hrm}[1]{{#1}\sp{\operatorname{H}}}

% Vanha transponointi; jäljellä yhteensopivuussyistä, ei syytä käyttää.
%\newcommand{\tr}{{}^{\text T}}

% Arcus- ja area-funktiot, jossa päähaara osoitetaan nimen päälle
% vedetyllä vaakasuoralla viivalla (alkaa olla vanhentunutta,
% voitaisiin luopua):

%\newcommand{\arccot}{\operatorname{arccot}}
%\newcommand{\asin}{\operatorname{\overline{arc}sin}}
%\newcommand{\acos}{\operatorname{\overline{arc}cos}}
%\newcommand{\atan}{\operatorname{\overline{arc}tan}}
%\newcommand{\acot}{\operatorname{\overline{arc}cot}}

%\newcommand{\arsinh}{\operatorname{arsinh}}
%\newcommand{\arcosh}{\operatorname{arcosh}}
%\newcommand{\artanh}{\operatorname{artanh}}
%\newcommand{\arcoth}{\operatorname{arcoth}}
%\newcommand{\acosh}{\operatorname{\overline{ar}cosh}}

% Signum, syt, pyj:

%\newcommand{\sg}{\operatorname{sgn}}
%\renewcommand{\gcd}{\operatorname{syt}}
%\newcommand{\lcm}{\operatorname{pyj}}

% Lyhennemerkintöjä: derivaatta, osittaisderivaatta, gradientti,
% derivaattaoperaattori, vektorin komponentti, integraalin ylä- ja
% alasumma, Suomessa (ja Saksassa?) käytetty integraalin sijoitus-
% merkintä, integraali (rakenteellinen määrittely):

%\newcommand{\der}[2]{\frac{\dd #1}{\dd #2}}
%\newcommand{\osder}[2]{\frac{\partial #1}{\partial #2}}
%\newcommand{\grad}{\operatorname{grad}}
%\newcommand{\Df}{\operatorname{D}} 
%\newcommand{\comp}{\operatorname{comp}}
%\newcommand{\ys}[1]{\overline S_{#1}}
%\newcommand{\as}[1]{\underline S_{#1}}
%\newcommand{\sijoitus}[2]{\biggl/_{\null\hskip-6pt #1}^{\null\hskip2pt #2}} 
%\newcommand{\itgr}[4]{\int_{#1}^{#2}#3\,\dd #4}

% Matriiseja, joille voidaan antaa alkioiden sijoittamista sarakkeen
% vasempaan tai oikeaan reunaan tai keskelle osoittava lisäparametri
% (l, r tai c); ympärillä kaarisulut, hakasulut, pystyviivat (determinantti)
% tai ei mitään;
% esimerkiksi \begin{cmatrix}{ll}1 & -1 \\ -1 & 1 \end{cmatrix}:

%\newenvironment{cmatrix}[1]{\left(\begin{array}{#1}}{\end{array}\right)}
%\newenvironment{Cmatrix}[1]{\left[\begin{array}{#1}}{\end{array}\right]}
%\newenvironment{dmatrix}[1]{\left|\begin{array}{#1}}{\end{array}\right|}
%\newenvironment{ematrix}[1]{\begin{array}{#1}}{\end{array}}

% Kaunokirjoitussymboli:

%\newcommand{\Cal}{\mathcal}

% Isokokoinen summa:

%\newcommand{\dsum}[2]{{\displaystyle \sum_{#1}^{#2}}}

% Tuplaintegraali umpinaisen pinnan yli; korvataan jos parempi löytyy:
%\newcommand\oiint{\begingroup
% \displaystyle \unitlength 1pt
% \int\mkern-7.2mu
% \begin{picture}(0,3)
%   \put(0,3){\oval(10,8)}
% \end{picture}
% \mkern-7mu\int\endgroup}
       % Haetaan joitakin makroja

% Kieli:
% Kielesi, jolla kandidaatintyön kirjoitat: finnish, swedish, english.
% Tästä tulee mm. tietyt otsikkonimet ja kuva- ja taulukkoteksteihin
% (Kuva, Figur, Figure), (Taulukko, Tabell, Table) sekä oikea tavutus.
\selectlanguage{finnish}
%\selectlanguage{swedish}
%\selectlanguage{english}

% Sivunumeroinnin kanssa pieniä ristiriitaisuuksia.
% Toimitaan pääosin lähteen "Kirjoitusopas" luvun 5.2.2 mukaisesti.
% Sivut numeroidaan juoksevasti arabialaisin siten että
% ensimmäiseltä nimiölehdeltä puuttuu numerointi.
\pagestyle{plain}
\pagenumbering{arabic}
% Muita tapoja: kandiohjeet: ei numerointia lainkaan ennen tekstiosaa
%\pagestyle{empty}
% Muita tapoja: kandiohjeet: roomalainen numerointi alussa ennen tekstiosaa
%\pagestyle{plain}
%\pagenumbering{roman}        % i,ii,iii, samalla alustaa laskurin ykköseksi

% ---------------------------------------------------------------------
% -------------- Luettelosivut alkavat --------------------------------
% ---------------------------------------------------------------------

% -------------- Nimiölehti ja sen tiedot -----------------------------
%
% Nimiölehti ja tiivistelmä kirjoitetaan seminaarin mukaan joko
% suomeksi tai ruotsiksi (ellei erityisesti kielenä ole englanti).
% Tiivistelmän voi suomen/ruotsin lisäksi kirjoittaa halutessaan
% myös englanniksi. Eli tiivistelmiä tulee yksi tai kaksi kpl.
%
% "\MUUTTUJA"-kohdat luetaan aaltosci_t.sty:ä varten.

\author{Atte Keinänen}


% Otsikko nimiölehdelle. Yleensä sama kuin seuraavana oleva \TITLE,
% mutta jos nimiölehdellä tarvetta "kaksiosaiselle" kaksiriviselle
\title{Laiska evaluointi funktionaalisessa ohjelmoinnissa}
% 2-osainen otsikko:
%\title{\LaTeX{}-pohja kandidaatintyölle \\[5mm] Pitkiä rivejä kokeilun vuoksi.}

% Otsikko tiivistelmään. Jos lisäksi engl. tiivistelmä, niin viimeisin:
\TITLE{Laiska evaluointi funktionaalisessa ohjelmoinnissa}
%\TITLE{\LaTeX{} för kandidatseminariet med jättelång rubrik som fortsätter och
% fortsätter ännu}
\ENTITLE{\LaTeX{} template for Bachelor thesis with a pretty long title %
line which continues ynd continues}
% 2-osainen otsikko korvataan täällä esim. pisteellä:
%\TITLE{\LaTeX{}-pohja kandidaatintyölle. Pitkiä rivejä kokeilun vuoksi.}

% Ohjaajan laitos suomi/ruotsi ja tarvittaessa eng (tiivistelmän kieli/kielet)
\DEPT{Tietotekniikan laitos}
% suomi:
%\DEPT{Tietotekniikan laitos}               % T
%\DEPT{Tietojenkäsittelytieteen laitos}     % TKT
%\DEPT{Mediatekniikan laitos}               % ME
% ruotsi:
%\DEPT{Institutionen för datateknik}        % T
%\DEPT{Institutionen för datavetenskap}     % TKT
%\DEPT{Institutionen för mediateknik}       % ME
% englanti:
%\ENDEPT{Department of Computer Science Engineering}     % T
%\ENDEPT{Department of Information and Computer Science} % TKT
%\ENDEPT{Department of Media Technology}                 % ME

% Vuosi ja päivämäärä, jolloin työ on jätetty tarkistettavaksi.
\YEAR{2016}
\DATE{\today}
%\DATE{31. helmikuuta 2011}
%\DATE{Den 31 februari 2011}
\ENDATE{MonthName 31, 20xx}

% Kurssin vastuuopettaja ja työsi ohjaaja(t)
\SUPERVISOR{Professori Juho Rousu}
\INSTRUCTOR{Yliopistonlehtori Juha Sorva}
%\INSTRUCTOR{Ohjaajantitteli Sinun Ohjaajasi, ToinenTitt Matti Meikäläinen}
% DI       // på svenska DI diplomingenjör
% TkL      // TkL teknologie licentiat
% TkT      // TkD teknologie doctor
% Dosentti Dos. // Doc. Docent
% Professori Prof. // Prof. Professor
%
% Jos tiivistelmä englanniksi, niin:
\ENSUPERVISOR{TitleOfResponsibleTeacher NameofResponsibleTeacher}
\ENINSTRUCTOR{Your instructor, titleOfInstructor}
% M.Sc. (Tech)  // M.Sc. (Eng)
% Lic.Sc. (Tech)
% D.Sc. (Tech)   // FT filosofian tohtori, PhD Doctor of Philosophy
% Docent
% Professor

% Kirjoita tänne HOPS:ssa vahvistettu pääaineesi.
% Pääainekoodit TIK-opinto-oppaasta.

\PAAAINE{Informaatioverkostot}
\CODE{SCI3026}

%\PAAAINE{Ohjelmistotuotanto ja -liiketoiminta}
%\CODE{T3003}
%
%\PAAAINE{Tietoliikenneohjelmistot}
%\CODE{T3005}
%
%\PAAAINE{WWW-teknologiat} % vuodesta 2010
%\CODE{IL3012}
%
%\PAAAINE{Mediatekniikka} % vuoteen 2010, kts. seur.
%\CODE{T3004}
%
%\PAAAINE{Mediatekniikka} % vuodesta 2010, kts. edell.
%\CODE{IL3011}
%
%\PAAAINE{Tietojenkäsittelytiede} % vuodesta 2010
%\CODE{IL3010}
%
%\PAAAINE{Informaatiotekniikka} % vuoteen 2010
%\CODE{T3006}
%
%\PAAAINE{Tietojenkäsittelyteoria} % vuoteen 2010
%\CODE{T3002}
%
%\PAAAINE{Ohjelmistotekniikka}
%\CODE{T3001}

% Avainsanat tiivistelmään. Tarvittaessa myös englanniksi:

\KEYWORDS{laiska evaluointi, evaluointistrategia, funktionaalinen ohjelmointi}
\ENKEYWORDS{key, words, the same as in FIN/SWE}

% Tiivistelmään tulee opinnäytteen sivumäärä.
% Kirjoita lopulliset sivumäärät käsin tai kokeile koodia.
%
% Ohje 29.8.2011 kirjaston henkilökunnalta:
%   - yhteissivumäärä nimiölehdeltä ihan loppuun
%   - "kaikkien yksinkertaisin ja yksiselitteisin tapa"
%
% VANHA // Ohje 14.11.2006, luku 4.2.5:
% VANHA // - sivumäärä = tekstiosan (alkaen johdantoluvusta) ja
% VANHA //  lähdeluettelon sivumäärä, esim. "20"
% VANHA // - jos liitteet, niin edellisen lisäksi liitteiden sivumäärä,
% VANHA //  tyyli "20 + 5", jossa 5 sivua liitteitä
% VANHA // - HUOM! Tässä oletuksena sivunumerointi alkaa nimiölehdestä
% VANHA //  sivunumerolla 1. %   Toisin sanoen, viimeisen lähdeluettelosivun
% VANHA //  sivunumero EI ole sivujen määrä vaan se pitää laskea tähän käsin

\PAGES{Täydennetään myöhemmin}
%\PAGES{23}  % kaikki sivut laskettuna nimiölehdestä lähdeluettelon tai
             % mahdollisten liitteiden loppuun. Tässä 23 sivua

%\thispagestyle{empty}  % nimiölehdellä ei ole sivunumerointia; tyylin mukaan ei tehdäkään?!

\maketitle             % tehdään nimiölehti

% -------------- Tiivistelmä / abstract -------------------------------
% Lisää abstrakti kandikielellä (ja halutessasi lisäksi englanniksi).

% Edelleen sivunumerointiin. Eräs ohje käskee aloittaa sivunumeroiden
% laskemisen nimiösivulta kuitenkin niin, että sille ei numeroa merkitä
% (Kauranen, luku 5.2.2). Näin ollen ensimmäisen tiivistelmän sivunumero
% on 2. \maketitle komento jotenkin kadottaa sivunumeronsa.
\setcounter{page}{2}    % sivunumeroksi tulee 2

% Toistaiseksi tarpeeton
% % done, pitäis olla kunnossa
% --------------------------------------------------------------------
% Cheatsheet:
% \verb!lazy val!   -- esim. lyhyisiin koodinpätkiin
% \begin{verbatim}  -- esim. pitkiin koodinpätkiin
% `` tekstiä ''     -- lainausmerkit
% \textbf{huom!}    -- boldaus
% \begin{quotation}
% \noindent \it     -- quoten lisäys
% \ldots            -- kolme pistettä
% footnote{juh}     -- alaviite
% \citet, \citep
% \citet[s.234]{}   -- viitteet, http://merkel.zoneo.net/Latex/natbib.php
% \begin{sloppypar} -- ahdas teksti
% \clearpage        -- onkkelmien korjaamiseen lukujen kanssa
% --                -- yhdysviiva
% \enumerate
% \itemize          -- listat
% \begin[htb]{figure}
% \begin[htb]{table} - Kuva ja taulukko, (h)ere, (t)op, (b)ottom
% \begin{equation}  -- kaava
% \label{eq:kaava1} -- laabeli
%!TEX root = main.tex

% Tiivistelmät tehdään viimeiseksi.
%
% Tiivistelmä kirjoitetaan käytetyllä kielellä (JOKO suomi TAI ruotsi)
% ja HALUTESSASI myös samansisältöisenä englanniksi.
%
% Avainsanojen lista pitää merkitä main.tex-tiedoston kohtaan \KEYWORDS.

\begin{fiabstract}
% Tiivistelmän tyypillinen rakenne:
% (1) aihe, tavoite ja rajaus
% (heti alkuun, selkeästi ja napakasti, ei johdattelua);
% (2) aineisto ja menetelmät (erittäin lyhyesti);
% (3) tulokset (tälle enemmän painoarvoa);
% (4) johtopäätökset (tälle enemmän painoarvoa).
Laiska evaluointi on ohjelmointikielten suoritusta ohjaava periaate, jossa ajatuksena on turhan ja liian aikaisen lausekkeiden evaluoinnin välttäminen. Selvitin työssäni laiskan evaluoinnin yhteydessä käytettävää käsitteistöä, laiskan evaluoinnin historiaa ja sen nykysovellutuksia. Lisäksi selvitin, millaisia etuja ja heikkouksia laiskaa evaluointiin liittyy.

Työn menetelmänä on narratiivinen kirjallisuuskatsaus. Aineistona käytin akateemisia artikkeleita, oppikirjoja, blogikirjoituksiaja ohjelmointikielten dokumentaatioita. Etuja ja heikkouksia selvitin myös haastattelujen avulla.

Käsitteistöä selvittäessäni päädyin siihen, että laiska evaluointi on yläkäsite usealle eri evaluointisemantiikalle. Evaluointisemantiikat määrittävät, milloin ja miten funktion parametrilausekkeet evaluoidaan. Historian osalta tuli selville, että laiskan evaluoinnin kehitys nivoutuu tiiviisti Haskell-ohjelmointikielen tutkimukseen, ja että Haskell on ainoa yleisessä käytössä oleva kieli, joka käyttää laiskaa evaluointia laajasti. Totesin, että nykysovellutuksia on useita erilaisia, ja niitä on käytössä kaikissa suosituissa ohjelmointikielissä ja niiden apukirjastoissa. Näitä sovellutuksia ovat muun muassa laiskat listat ja generaattorit. Etuja ja heikkouksia tarkastellessa tuli esille, että Haskellissa laiska evaluointi vaikeuttaa sen ajonaikaisen suorituksen seuraamista ja aiheuttaa ajoittaisia muistivuotoja. Toisaalta rajatummista sovellutuksista oli hyviä kokemuksia.

Laiska evaluointi on ollut suuressa roolissa etenkin funktionaalisen ohjelmointityylin kehittymisessä. Nykyään laiska evaluointi vaikuttaa väistyvän ohjelmointikielien suunnittelua määräävänä periaatteena, mutta sitä osataan käyttää hyödyksi sovellutuksissa, joissa laajemman käytön aiheuttamia ongelmia ei tule.


%
%Tiivistelmätekstiä tähän (\languagename). Huomaa, että tiivistelmä tehdään %vasta kun koko työ on muuten kirjoitettu.
\end{fiabstract}

%\begin{svabstract}
%  Ett abstrakt hit
%%(\languagename)
%\end{svabstract}

%\begin{enabstract}
% Here goes the abstract
%%(\languagename)
%\end{enabstract}

% \newpage                       % pakota sivunvaihto

% -------------- Sisällysluettelo / TOC -------------------------------

\tableofcontents

\label{pages:prelude}
\clearpage                     % kappale loppuu, loput kelluvat tänne, sivunv.
%\newpage

% -------------- Symboli- ja lyhenneluettelo -------------------------
% Lyhenteet, termit ja symbolit.
% Suositus: Käytä vasta kun paljon symboleja tai lyhenteitä.
%
%% -------------- Symbolit ja lyhenteet --------------
%
% Suomen kielen lehtorin suositus: vasta kun noin 10-20 symbolia
% tai lyhennettä, niin käytä vasta sitten.
%
% Tämä voi puuttuakin. Toisaalta jos käytät paljon akronyymejä,
% niin ne kannattaa esitellä ensimmäisen kerran niitä käytettäessä.
% Muissa tapauksissa lukija voi helposti tarkistaa sen tästä
% luettelosta. Esim. "Automaattinen tietojenkäsittely (ATK) mahdollistaa..."
% "... ATK on ..."

\addcontentsline{toc}{section}{Käytetyt symbolit ja lyhenteet}

\section*{Käytetyt symbolit ja lyhenteet}
%?? Käytetyt lyhenteet ja termit ??
%?? Käytetyt lyhenteet / termit / symbolit ??
%\section*{Abbreviations and Acronyms}

\begin{center}
\begin{tabular}{p{0.2\textwidth}p{0.65\textwidth}}
3GPP  & 3rd Generation Partnership Project; Kolmannen sukupolven 
matkapuhelupalvelu \\ 
ESP & Encapsulating Security Payload; Yksi IPsec-tietoturvaprotokolla \\ 
$\Omega_i$ & hilavitkuttimen kulmataajuus \\
$\mathbf{m}_{ic}$ & hilavitkutinjärjestelmän $i$ painokertoimet \\
\end{tabular}
\end{center}

\vspace{10mm}

Tähän voidaan listata kaikki työssä käytetyt lyhenteet. Lyhenteistä
annetaan selityksenä sekä alkukielinen termi kokonaisuudessaan
(esim. englanninkielinen lyhenne avattuna sanoiksi) että sama
suomeksi. Jos suoraa käännöstä ei ole tai sellaisesta on vaikea saada
sujuvaa, voi käännöksen sijaan antaa selityksen siitä, mitä kyseinen
käsite tarkoittaa. Jos lyhenteitä ei esiinny työssä paljon, ei tätä
osiota tarvita ollenkaan. Yleensä luettelo tehdään, kun lyhenteitä on
10--20 tai enemmän. Vaikka lyhenteet annettaisiinkin tässä
keskitetysti, ne pitää silti avata sekä suomeksi että alkukielellä
myös itse tekstissä, kun ne esiintyvät siellä ensi kertaa.  Käytetyt
lyhenteet -osion voi nimetä myös ``Käytetyt lyhenteet ja termit'', jos
luettelossa on sekä lyhenteitä että muuta käsitteenmäärittelyä.

\textbf{TIK.kand suositus: Lisää lyhenne- tai symbolisivu, kun se
  näyttää luontevalta ja järkevältä. (Käytä vasta kun lyhenteitä yli 10.)}

%Jos tarvitset useampisivuista taulukkoa, kannattanee käyttää 
%esim. \verb!supertabular*!-ympäristöä, josta on kommentoitu esimerkki
%toisaalla tekstiä.



%\clearpage                     % luku loppuu, loput kelluvat tänne
\newpage

% -------------- Kuvat ja taulukot ------------------------------------
% Kirjoissa (väitöskirja) on usein tässä kuvien ja taulukoiden listaus.
% Suositus: Ei kandityöhön.

% -------------- Alkusanat --------------------------------------------
% Suositus: ÄLÄ käytä kandidaatintyössä. Jos käytät, niin omalle
% sivulleen käyttäen tarvittaessa \newpage
%
%% --------------- Alkusanat -------------------------------------------
%
% Suositus: Älä käytä kandidaatintyössä.
%

\addcontentsline{toc}{section}{Alkusanat}

\section*{Alkusanat}
%\section*{Förord}
%\section*{Acknowledgements}

Alkusanoissa voi kiittää tahoja, jotka ovat merkittävästi edistäneet
työn valmistumista. Tällaisia voivat olla esimerkiksi yritys, jonka
tietokantoja, kontakteja tai välineistöä olet saanut käyttöösi,
haastatellut henkilöt, ohjaajasi tai muut opettajat ja myös
henkilökohtaiset kontaktisi, joiden tuki on ollut korvaamatonta työn
kirjoitusvaiheessa. Alkusanat jätetään tyypillisesti pois
kandidaatintyöstä, joka on laajuudeltaan vielä niin suppea, ettei
kiiteltäviä tahoja luontevasti ole.

\textbf{TIK.kand suositus: Älä käytä tällaista lukua.}

\vskip 10mm
Espoossa 31. helmikuuta 2011
\vskip 15mm
Teemu Teekkari


%\clearpage                     % luku loppuu, loput kelluvat tänne
%\newpage                       % pakota sivunvaihto
%
%SH: Alkusanoissa voi kiittää tahoja, jotka ovat merkittävästi edistäneet
% työn valmistumista. Tällaisia voivat olla esimerkiksi yritys, jonka
% tietokantoja, kontakteja tai välineistöä olet saanut käyttöösi,
% haastatellut henkilöt, ohjaajasi tai muut opettajat ja myös
% henkilökohtaiset kontaktisi, joiden tuki on ollut korvaamatonta työn
% kirjoitusvaiheessa. Alkusanat jätetään tyypillisesti pois
% kandidaatintyöstä, joka on laajuudeltaan vielä niin suppea, ettei
% kiiteltäviä tahoja luontevasti ole.

% ---------------------------------------------------------------------
% -------------- Tekstiosa alkaa --------------------------------------
% ---------------------------------------------------------------------

% Muutetaan tarvittaessa ala- ja ylätunnisteet
%\pagestyle{headings}          % headeriin lisätietoja
%\pagestyle{fancyheadings}     % headeriin lisätietoja
%\pagestyle{plain}             % ei header, footer: sivunumero

% Sivunumerointi, jos käytetty 'roman' aiemmin
% \pagenumbering{arabic}        % 1,2,3, samalla alustaa laskurin ykköseksi
% \thispagestyle{empty}         % pyydetty ensimmäinen tekstisivu tyhjäksi

% input-komento upottaa tiedoston
% --------------------------------------------------------------------
% Cheatsheet:
% \verb!lazy val!   -- esim. lyhyisiin koodinpätkiin
% \begin{verbatim}  -- esim. pitkiin koodinpätkiin
% `` tekstiä ''     -- lainausmerkit
% \textbf{huom!}    -- boldaus
% \begin{quotation}
% \noindent \it     -- quoten lisäys
% \ldots            -- kolme pistettä
% footnote{juh}     -- alaviite
% \citet, \citep
% \citet[s.234]{}   -- viitteet, http://merkel.zoneo.net/Latex/natbib.php
% \begin{sloppypar} -- ahdas teksti
% \clearpage        -- onkkelmien korjaamiseen lukujen kanssa
% --                -- yhdysviiva
% \enumerate
% \itemize          -- listat
% \begin[htb]{figure}
% \begin[htb]{table} - Kuva ja taulukko, (h)ere, (t)op, (b)ottom
% \begin{equation}  -- kaava
% \label{eq:kaava1} -- laabeli
%!TEX root = main.tex

\section{Johdanto}

% Määriteltäviä käsitteitä:
% Non-Strict Semantics
% Jotain muuta

% 1. ja 2. kirjoitetaan lopulliseen muotoonsa sitten, kun muut osat ovat jo hyvällä mallilla!

% 1. Aloitan tarinan todellisen elämän esimerkillä, jossa on tullut vastaan aito tarve sille, että koodia ei evaluoida tiukasti "rivi kerrallaan". Siisteintä (ja emotionaalisesti tehokkainta) olisi saada oikein kouriintuntuva esimerkki tilanteesta, jossa ollaan joduttu ongelmiin, kun laiskaa evaluointia ei ole aluksi käytetty. Esimerkillä pyrin herättämään lukijan mielenkiinnon, ja nivomaan aiheen välittömästi konkretiaan ennen siihen syventymistä.

Laiska evaluointi on 1970-luvulta juontuva ohjelmoinnin konsepti, jonka ytimessä on turhan ja liian aikaisen laskennan välttäminen ohjelmakoodia suorittaessa. Eli jos esimerkiksi listan alkiota ei koskaan tarvita, niin sen arvoakaan ei ole syytä laskea. Vastaavasti jos arvoa tarvitaankin, sitä ei lasketa listaa muodostettaessa, vaan vasta sitten kuin sitä on tarve käyttää.

Laiska evaluointi on nimityksenä vakiintunut, mutta ``laiskuuden'' osalta hieman harhaanjohtava. Paremmin konseptia kuvaa sen englanninkielinen synonyymi \textit{call-by-need evaluation}, joka korostaa sitä, että laskentaa tehdään vasta kun tietoa tarvitaan. Toinen tulkinta on, että laskentaa viivytetään niin kauan, kunnes arvo on pakko laskea.

% 2. Kerron, kuinka 2000-luvulla on näkynyt paljon kehitystä, jossa ajatusta siitä, että lausekkeita ei ole aina pakko evaluoida välittömästi kun ne koodissa määritellään, on sovellettu laajemmalle. Esimerkiksi uudet ohjelmointikielet kuten Scala tukevat sitä, ja monet laiskan evaluoinnin ajatusta hyödyntävät kirjastot ovat nähneet päivänvalon esimerkiksi JavaScriptissä. Jos tuntuu luontevalta, niin samassa yhteydessä voi antaa lukijalle alustavan intuition laiskasta evaluoinnista vertauskuvia hyödyntäen (korostaen että kyseessä analogia, ei eksakti käsitemäärittely).

Laiskan evaluoinnin tekee aiheena kiinnostavaksi se, että sitä hyödynnetään kasvavissa määrin niin ohjelmointikielien suunnittelussa kuin yksittäisissa sovelluskirjastoissa. Esimerkiksi uudehkot ohjelmointikielet, kuten Scala ja Clojure, tarjoavat erityisiä syntaktisia rakenteita sitä varten. Toisaalta myös monet suositut apukirjastot, esimerkiksi JavaScript-kielen Immutable.js, Lazy.js ja Bacon.js, hyödyntävät sen periaatteita.

Laiskaa evaluointia ei ole yhtä tapaa tehdä ``oikein'', vaan käyttökohteiden ja toteutustapojen diversiteetti on suuri. Myös tavoiteltavat hyödyt ovat erilaisia. Verrataan esimerkiksi ohjelmointikielen ja apukirjaston laatijoiden tavoitteita: ohjelmointikielen laatija voi käyttää laiskaa evaluointia kielen ilmaisuvoiman kasvattamiseen, kun taas apukirjaston laatija voi päästä sen avulla kilpailevia kirjastoja parempaan suorituskykyyn.

Ymmärrän tässä työssä laiskan evaluoinnin laajasti, ja pyrin luomaan kokonaiskuvaa konseptin käyttökohteista ja toteutustavoista.  Seuraavaksi määrittelen tarkemmin, mitä tarkoitan laiskalla evaluoinnilla, ja mitä muita käsitteitä konseptiin tiiviisti liittyy. Sen jälkeen esittelen tutkimuskysymykset, joihin työni pyrkii vastaamaan.

% 3. Määrittelen sitä tietotekniikan kontekstia, jossa nyt liikutaan, ja määrittelen keskeisiä käsitteitä "up-to-bottom". Ensiksi kerron mistä koodin evaluoinnissa on kyse, millaisia erilaisia evaluointitapoja on olemassa (vielä kuitenkaan vertailematta niiden vahvuuksia/heikkouksia). Lopulta tarkennan laiskaan evaluointiin sekä siihen tiiviisti liittyviin käsitteisiin (mm. call-by-need, call-by-name ja thunk). Lukija saa tästä tarvittavaa sanavarastoa tekstin lukemiseen.

\subsection{Keskeisten käsitteiden määrittely}

\textit{Lauseke} tarkoittaa ohjelmointikielen ilmaisua, jolle voidaan määrittää arvo. Esimerkiksi aritmeettiset operaatiot, muuttujien nimet ja funktiokutsut ovat lausekkeita.

\textit{Evaluointi} tarkoittaa tarkoittaa lausekkeen arvon laskemista.

\textit{Evaluointistrategia} tarkoittaa ohjelmointikielen sääntöjä, jotka määrittävät, missä tilanteissa lausekkeita evaluoidaan.

\textit{Laiska evaluointi} tarkoittaa evaluointistrategiaa, joka viivyttää lausekkeen evaluointia saakka, kunnes sitä tarvitaan \citep{watt2004programming}. Käsitettä voi käyttää kuvaamaan myös yksittäistä lauseketta abstraktimman operaation, kuten esimerkiksi kokoelmien käsittelyn, suorittamisen viivästämistä vastaavaan tapaan.

% 4. Kerron, että olen hakemassa vastausta kahteen tutkimuskysymykseen: mikä sen merkitys tietotekniikassa on nykyhetkellä, ja millaisia vahvuuksia tai heikkouksia laiskalla evaluoinnilla on. Samassa yhteydessä kerron perustellen, että olen tarkoituksella jättänyt laiskan evaluoinnin matemaattisen ja kääntäjäteknisen tarkastelun tämän työn ulkopuolelle. Myös työn muista tavoitteista mainitseminen on mahdollista: voin vaikka mainita siitä, että toivon työn tulosten olevan hyödynnettävissä ohjelmoinnin opetuksessa.

% 5. Kerron käyttämäni tieteelliset menetelmät (kirjallisuuskatsaus ja mielipidekartoitus) ja perustelen miksi päädyin juuri näihin menetelmiin. Valotan kirjallisuuskatsauksen prosessia: mitä hakusanoja käytin, mitä tietokantoja hyödynsin ja millaisia määriä artikkeleita kävin läpi. Käyn samaan tapaan läpi mielipidekartoituksen toteutuksen, eli sekä tunnettujen kirjoittajien blogien läpikäynnin että kokemusten keräämisen tuttavapiiristäni.

\subsection{Tutkimuskysymykset}

Haluan saada kartoitettua työlläni laiskan evaluoinnin käyttökelpoisuutta tutkimuksessa ja työelämässä, ja haluan myös tuottaa sellaista tietoa, joka olisi käyttökelpoista laiskan evaluoinnin konseptien yliopisto-opetuksen suunnittelutyössä. Päädyin näiden tavoitteiden kautta seuraaviin tutkimuskysymyksiin:
\begin{enumerate}
  \item{Mikä on laiskan evaluoinnin merkitys nykypäivänä?}
  \item{Mitkä ovat laiskan evaluoinnin hyödyt ja haitat?}
\end{enumerate}

Kysymyksessä 1 laiskan evaluoinnin merkityksen tutkiminen tarkoittaa, että tarkastelen (a) laiskan evaluoinnin varhaista historiaa pohjustuksenomaisesti, (b) laiskan evaluoinnin leviämistä moderneihin ohjelmointikieliin sekä kielien apukirjastoihin, (c) laiskan evaluoinnin kautta kehittyneitä uusia ohjelmoinnin konsepteja ja (d) laiskaa evaluointia käyttävien teknologioiden hyödyntämistä tutkimuksessa ja työelämässä.

Kysymyksessä 2 selvitän sitä, millaisia subjektiivisia mielipiteitä laiskaa evaluointia hyödyntäviä teknologioita käyttävillä ihmisillä on sekä laiskasta evaluoinnista yleisesti, että kysymyksen 1 tarkastelussa esiin tulleista ohjelmointikielistä ja apukirjastoista.


\subsection{Tutkimusmenetelmät}

Tarkastelen kysymystä 1 narratiivisen kirjallisuuskatsauksella avulla. Tavoitteena on kartoittaa, millaista tietoa aiheesta tällä hetkellä on, ja tuoda sitä yhteen helppotajuisen narratiivin muotoon. Tällä tavoin luotu katsaus soveltuu hyvin materiaaliksi opettajille \citep[s. 312]{baumeister1997writing}, joten menetelmävalinta on linjassa sen tavoitteen kanssa, että työstä olisi hyötyä yliopisto-opetuksen suunnittelussa.

Kirjallisuuskatsauksen materiaalina käytän ensisijaisesti akateemisia artikkeleita, joita aiheesta on kirjoitettu paljon. Hain artikkeleita ensisijaisesti Scopus-tietokannasta, jossa käytin seuraavaa hakulauseketta:

\begin{verbatim}
 ("lazy evaluation" OR "non-strict" OR "call-by-need" OR "call by need")
 AND ("functional")
\end{verbatim}

And-operaattorin vasemmalla puolella on kaikki laiskan evaluoinnin tyypillisimmät synonyymit vaihtoehtoisina hakutermeinä.\footnote{Puhtaan funktionaalisen ohjelmoinnin, erityisesti Haskell-ohjelmointikielen, sanastossa käsitteellä \textit{non-strict} on usein laiskasta evaluoinnista eriävä merkitys. Tätä kasitellään tarkemmin luvussa 2. Kirjallisuudessa käsitteet kuitenkin esiintyvät myös toistensa synonyymeina.} Operaattorin oikealla puolella oleva hakutermi \textit{functional} rajaa hakutulokset funktionaalisen ohjelmoinnin piiriin. Tämä hakulauseke tuotti otsikosta, tiviistelmästä ja avainsanoista haettaessa 326 tulosta. Tämä oli sopiva määrä tuloksia tutkimuskysymysten kannalta relevanteimpien artikkeleiden valikointia ajatellen.

Scopus-tietokannan lisäksi seuloin artikkeleita samalla hakutermillä myös Google Scholarista. Lisäksi Stack Overflow -kysymyspalvelun vastausten ja Haskellin oman wiki-alustan kautta löytyi useita linkkejä relevantteihin artikkeleihin. Täydensin akateemisista artikkeleista saatua tietoa myös vähäisissä määrin blogikirjoituksilla. Niistä on iloa sellaisten uusien kehityskulkujen esittelyssä, joiden esiintyminen akateemisissa artikkeleissa on hyvin vähäistä.

\textit{Pitää vielä selvittää, mitkä tutkimusmenetelmät kuvaisivat parhaiten sitä mielipideselvitystä, joka tehdään tutkimuskysymykseen 2 vastaamiseksi.}
% 6. Esittelen työn rakennetta perustellen sen menetelmien ja tutkimuskysymysten avulla. Kerron, että käsittelen ensin katsausmaisesti laiskan evaluoinnin varhaista historiaa, minkä jälkeen poraudun ensimmäiseen tutkimuskysymykseen laiskasta evaluoinnista nykypäivänä. Sitten, kun lukijalla on hyvä yleiskuva nykypäivän sovellutuksista, on sujuvaa siirtyä puhumaan yksittäisten sovellutusten yksityiskohdista sekä vahvuuksista ja heikkouksista.

Kirjallisuuskatsauksen ja mielipideselvityksen käsittelyjärjestystä tässä työssä esittelee seuraava osio, työn rakenne.

\subsection{Työn rakenne}

\textit{Kirjoitan myöhemmin auki, kun sisällysluettelo hahmottuu:}

Esittelen työn rakennetta perustellen sen menetelmien ja tutkimuskysymysten avulla. Kerron, että käsittelen ensin katsausmaisesti laiskan evaluoinnin varhaista historiaa, minkä jälkeen poraudun ensimmäiseen tutkimuskysymykseen laiskasta evaluoinnista nykypäivänä. Sitten, kun lukijalla on hyvä yleiskuva nykypäivän sovellutuksista, on sujuvaa siirtyä puhumaan yksittäisten sovellutusten yksityiskohdista sekä vahvuuksista ja heikkouksista.

% 7. Nykypäivän tilannetta taustoitan aluksi kertomalla, kuinka idea laiskasta evaluoinnista on saanut alkunsa akateemisena ideana, jota alettiin kokeilla erilaisissa ohjelmointikielissä, ja jopa prosessoriarkkitehtuureissa. Tästä siirryn Haskellin kehitykseen, ja siihen, kuinka laiskan evaluoinnin vaatimus osaltaan johti puhtaasti funktionaalisen ohjelmoinnin paradigman kehittymiseen.  Käsittelyssä tässä kappaleessa siis käytännössä noin vuodet 1975-2000. Tässä elävöitän tekstiä historia-anekdooteilla.

\section{Laiskan evaluoinnin historiaa}

Laiska evaluointi sai alkusysäyksensä 1970-luvulla. Sarja julkaisuja loi pohjaa ajatukselle laiskoista funktionaalisista kielistä työkaluna käytännönläheiseen ohjelmistokehitykseen. Ajatus esiteltiin ensimmäisenä matemaattisesti lamdakalkyylin, funktionaalisen ohjelmoinnin kannalta keskeisen matemaattisen teorian, näkökulmasta \citep{wadsworth1971semantics}. Viisi vuotta myöhemmin julkaistiin toisistaan riippumatta kolme artikkelia \citep{henderson1976lazy,friedman1976cuns,saslmanualturner}, joissa esiteltiin laiskaa evaluointia ohjelmoinnin perspektiivistä.

1980-luvulla vaihteessa oli uraauurtavaa kehitystä kohti ensimmäisiä laiskoja ohjelmointikieliä. Saman aikaan kehitettiin myös täysin uudenlaisia tietokoneita, jotka kilpailivat alan standardin, Von Neumannin arkkitehtuurin, kanssa. Kuitenkin pidemmän päälle osoittautui, ettei tarvetta erikoisille arkkitehtuureille ole, vaan hyvin laadituilla ohjelmointikielten kääntäjillä voidaan päästä hyviin lopputuloksiin myös Von Neuman -arkkitehtuuria hyödyntävissä tietokoneissa. \citep{hudak2007history}

\subsection{Ei-tiukka semantiikka ja graafireduktio}

Yhdistävää 1980-luvun vaihteessa kehitetyille laiskoille ohjelmointikielille oli, että niissä alettiin hyödyntämään sittemmin vakiintuneita \textit{ei-tiukan semantiikan} ja \textit{graafireduktion} periaatteita. Jos ohjelmointikieli perustuu ei-tiukalle semantiikalle, niin lausekkeella voi olla arvo, vaikka jollakin sen alilausekkeista (pienemmistä lausekkeista, joista lauseke koostuu) ei olisi arvoa. Vastaavasti \textit{tiukkaan semantiikkaan} perustuvat ohjelmointikielet toimivat siten, että jos joltakin alilausekkeelta puuttuu arvo, niin lausekkellakaan ei ole arvoa, ja tällöin lausekkeen evaluoinnin voi ajatella epäonnistuneen \citep[s. 523]{scott2009programming}.

Tiukkaa semantiikkaa käytetään yleisesti imperatiivisissa ohjelmointikielissä. Useimmissa suosituissa imperatiivisissa kielissä, esimerkiksi Pythonissa, tiukan semantiikan toteutus perustuu funktioiden evaluointijärjestykseen liittyviin sääntöihin. Evaluointi etenee ``sisimmät ensin'' -periaattella. Siinä funktion kaikki argumentit evaluoidaan ennen kuin funktiota kutsutaan, ja evaluointi etenee sisimmästä funktiokutsusta ulompia funktiokutsuja kohti.

Vastaavasti ei-tiukan semantiikan tyypillisin toteutus, jota tapaa esimerkiksi Haskellissa, perustuu ``uloimmat ensin'' -periaatteeseen. Siinä funktioiden evaluointi etenee uloimmasta funktiokutsusta kohti sisempiä, ja ainoastaan ne funktion argumentit, joita funktio todellisuudessa käyttää, evaluoidaan. Siten Haskellissa on mahdollista tehdä jopa sellaisia funktioita, joka ei tarvitse argumentteja lainkaan, ja siten palauttaa arvon riippumatta siitä, onko parametrilausekkeilla arvoa vai ei. \citep{haskellwikinonstrict}.

Taulukko \ref{table:python_haskell_semantics} demonstroi semantiikan vaikutusta funktioiden evaluointiin. Siinä on toiminallisuudeltaan vastaava koodi suoritetaan sekä Pythonilla että Haskellilla. Funktio \verb!noreturn! aiheuttaa ikuisen silmukan, minkä tähden se ei koskaan palauta arvoa. Haskellilla lausekkeen evaluointi palauttaa arvon, koska \verb!noreturn! -funktiota ei koskaan kutsuta.

\makeatletter
\preto{\@verbatim}{\topsep=0pt \partopsep=0pt }
\makeatother
\newpage
\begin{table}[th]
  \begin{center}
    \begin{tabular}{|p{0.5\textwidth}|p{0.40\textwidth}|}
      \hline
       Python, tiukka semantiikka& Haskell, ei-tiukka semantiikka \\
      \hline
      \footnotesize

      \begin{alltt}
def noreturn(x):
    while True:
        x = -x
    return x # not reached

def even(x):
  return x % 2 == 0

> any(even(n) for n in [3, 2, noreturn(6)])
\(\Rightarrow\) (ei palauta arvoa)
    \end{alltt}
      &\footnotesize\begin{alltt}
noreturn :: Integer -> Integer
noreturn x = negate (noreturn x)

> any . even . [3, 2, noreturn 6]
\(\Rightarrow\) True

\end{alltt}

      \textit{Katsotaan, haluanko tässä vielä käydä evaluoinnin välivaiheita läpi, vai viittaanko samaan esimerkkiin myöhemmin uudelleen.}\\
      \hline
    \end{tabular}
    \caption{Esimerkki kielen semantiikan vaikututuksesta evaluointijärjestykseen }
    \label{table:python_haskell_semantics}
  \end{center}
\end{table}

Graafireduktio on tapa ei-tiukan semantiikan toteuttamiseen. Se esittää lausekkeet verkon muodossa, mikä mahdollistaa toistuvien lausekkeiden jakamisen muiden lausekkeiden kesken \citep{hudak1989conception}. Esimerkiksi lausekkeen \verb!(1+2)*(1+2)! verkkoesityksessä lauseke \verb!(1+2)! pystytään jakamaan, minkä myötä sen arvo tarvitsee evaluoida vain kerran. % LÄHDE?

Ei-tiukan semantiikan ``uloimmat ensin'' -suoritusjärjestys ja graafireduktiossa tapahtuva lausekkeiden jakaminen luovat perustan laiskalle evaluoinnille. Lausekkeen arvoa ei lasketa ennen kuin on tarve, eikä sitä turhaan lasketa uudelleen, jos sitä käytetään myöhemmin uudelleen.

\subsection{Haskellin ja puhtaan funktionaalisen ohjelmoinnin kehitys}

\citet{hudak2007history} kuvaa, kuinka 1980-luvun puolivälissä laiskaa evaluointia hyödyntäviä ohjelmointikieliä alkoi olla ruuhkaksi asti. Useimmat kielistä soveltuivat vain kapeaan määrään käyttökohteita, ja niillä ei ollut riittävää määrää käyttäjiä suuren suosion saavuttamiseksi. Kuitenkin artikkelin kirjoittajat olivat tällöin sitä mieltä, että kielet muistuttivat ominaisuuksiltaan hyvin paljon toisiaan. Alkoi kehittyä ajatus siitä, että olisi hyvä luoda yksi, yleinen kieli, joka korvaisi kerralla monia aikaisempia kieliä.

Tämä johti Haskell-ohjelmointikielen kehityksen aloittamiseen. Siitä vastasi Haskell-komitea, jossa vaikutti monia aikaisempien laiskaa evaluointia hyödyntäneiden kielien suunnittelijoita. Komitea onnistui keräämään yhteen aikaisemmin erillään samaa aihepiiriä tutkineita, mikä kiihdytti laiskan evaluoinnin parissa tapahtunutta tieteellistä kehitystä. Alkuvaiheen kehitystyö oli onnistunutta, ja Haskellista kehittyi etenkin tietojenkäsittelytieteen akateemisen tutkimuksen parissa suosittu kieli.

Haskell ei määritelmällisesti ole laiska ohjelmointikieli, vaan jo Haskell 1.1 -raportissa \citep{yale1991report} se määriteltiin ainoastaan ei-tiukkaa semantiikkaa käyttäväksi kieleksi. Käytännössäkään kieli ei ole täysin laiska, vaan Haskellin kääntäjät tekevät useita nopeusoptimointeja suorittamalla tiettyjä osia koodista tiukkaan semantiikan säännöllä. Useimmat kielen rakenteet noudattavat kuitenkin oletusarvoisesti laiskan evaluoinnin periaatteita.

Laiskassa evaluoinnissa koodia ei suoriteta imperatiivisten kielien tapaan rivi riviltä, vaan suoritusjärjestys määräytyy tarpeen mukaan. IO-operaatioita, eli esimerkiksi näytölle tulostamista tai käyttäjän syötteen odottamista, ei pystytä suorittamaan halutussa järjestyksessä pelkkien funktiokutsujen avulla. Tästä seurasi, että Haskellista tuli \textit{puhtaasti funktionaalinen} ohjelmointikieli. Käsitteellä tarkoitetaan, että kielen funktiokonstruktiot ovat funktioita matemaattisessa mielessä: funktion kutsuminen samoilla argumenteilla palauttaa aina saman arvon, eikä funktioilla voi suorittaa IO-operaatioita tai muita ohjelman tilaa muuttavia operaatioita.

% Jotain mainintaa pysyvätilaisista muuttujista?

Laiskan evaluoinnin periaatteiden noudattaminen on johtanut moniin innovaatioihin funktionaalisen ohjelmoinnin saralla. Koska Haskell on puhtaasti funktionaalinen kieli, täytyi kehittää kokonaan uusia konsepteja, jotta esimerkiksi IO-operaatioiden suorittaminen halutussa järjestyksessä ja niiden ketjuttaminen toisiinsa onnistuu. Tunnetuimmaksi konseptiksi on noussut \textit{monadi}. Alkuperäisessä monadin konseptin esitelleessä artikkelissa sitä kutsutaan ``imperatiivisen funktionaalisen ohjelmoinnin'' työkaluksi, mikä kuvaa monadien mahdollistamia sekventiaalisia IO-operaatioita \citep{PeytonJones199371}.

Niin monadi kuin monet muutkin Haskelliin kehitetyt puhtaan funktionaalisen ohjelmoinnin konseptit ovat levinneet myös sellaisiin ohjelmointikieliin, jotka ovat evaluointistrategialtaan tiukkoja. Siten laiskan evaluoinnin vaalimisen Haskellissa voi ajatella olleen hyödyksi tietojenkäsittelytieteen kehitykselle laajemminkin.

\subsection{Päättymättömät tietorakenteet ja kontrollirakenteet}

Jo 1980-luvun varhaisessa tutkimuksessa käsiteltiin laiskan evaluoinnin mahdollistamia laiskoja tietorakenteita, etenkin ``päättymättömiä tietorakenteita''. Näitä ovat muun muassa päättymättömät listat, puurakenteet ja tapahtumavirrat. Ne ovat tyypillisesti rekursion avulla toteutettuja, ja laiskan evaluoinnin ansiosta tietorakenteessa tarvitsee mennä vain niin ``syvälle'' kuin on tarvetta.

\newpage

Eräs klassinen esimerkki on matematiikan äärettömiä lukujonojen kuvaaminen. Seuraavassa on Fibonaccin lukujono ilmaistuna rekursiivisesti Haskellilla. Huomionarvoista on, kuinka paljon koodi muistuttaa tapaa, jolla rekursiivinen lukujono ilmaistaisiin matematiikassa:

\begin{alltt}
fib 0 = 0
fib 1 = 1
fib n = fib (n-1) + fib (n-2)
\end{alltt}

Laiska evaluointi mahdollistaa myös omien kontrollirakenteiden kirjoittamisen. Esimerkiksi if-ehtolauseesta on Haskellissa helppoa tehdä oma versio, jossa ehdon täyttymisen perusteella evaluoidaan vain jompikumpi vaihtoehtoisista lausekkeista:

\begin{alltt}
% Käyttö: myIf condition onTrue onFalse
myIf :: Bool -> a -> a -> a
myIf True  x _ = x
myIf False _ y = y
\end{alltt}

% Tässä olisi ehkä kiva mainita toki siitä, että täsmäkielen rakentamiseen vaikuttaa toki monta muutakin tekijää.
Omien kontrollirakenteiden tuki helpottaa kieleen upotettujen \textit{täsmäkielten} rakentamista. Täsmäkielillä tarkoitetaan pieniä, tavanomaisesti deklaratiivisia kieliä, jotka ovat hyvin ilmaisuvoimaisia tietyssä ongelma-avaruudessa \citep{van2000domain}. Täsmäkielet toteutetaan usein ohjelmointikielen apukirjastoina, ja täsmäkieli on siinä tapauksessa yhdistelmä ohjelmointikielen valmiita ominaisuuksia ja apukirjastossa siihen laadittuja lisäyksiä.

Seuraavassa luvussa siirrytään tarkastelemaan sitä, millaisia sovellutuksia laiskalle evaluoinnille on löydetty erilaisissa ohjelmointikielissä ja ohjelmointiin liittyvissä \mbox{konteksteissa.}

% 8. Käyn läpi sitä, kuinka Haskell ja laiska evaluointi alkoi inspiroimaan monia uusia ohjelmointikieliä ja ohjelmointikielien kirjastoja. Tässä kohtaa keskityn nimenomaan siihen diversiteettiin, eli millaisissa erilaisissa muodoissa laiska evaluointi nykyään esiintyy, ja kuinka sitä käytetään myös tiukan evaluoinnin rinnalla. Käyn läpi paljon esimerkkejä laiskan evaluoinnin sovellutuksista. Jos esimerkkien määrä on suuri, se voi aiheuttaa haasteita luvun jäsentämiselle.

\section{Laiskan evaluoinnin sovellutuksia}


Laiskalle evaluoinnille on löydetty lukuisia erilaisia sovellutuksia. Tähän lukuun on koottu niistä merkittävimpiä. \textit{Kirjoitetaan tämä pohjustusteksti uudelleen, kun sisältö on selkeytynyt.}

\subsection{Tuki Scala-ohjelmointikielessä}

\textit{Scala} on sekä funktionaalista että imperatiivista ohjelmointia tukeva yleiskäyttöinen ohjelmointikieli. Scala-lähdekoodi kääntyy Java-tavukoodiksi, ja kieli on täysin yhteensopiva Java-kirjastojen kanssa. Scala on semantiikaltaan tiukka kieli, mutta evaluointijärjestystä on mahdollista muuttaa paikallisesti kahdella mekanismilla: \textit{laiskoilla muuttujilla} ja \textit{by-name-parametreilla}.

Laiska muuttuja luodaan lisäämällä \verb!lazy! -avainsana muuttujan määrittelyn alkuun. Kyseisen muuttujan arvon määrittävä lauseke evaluoidaan vasta sitten, kun muuttujaa käytetään ensimmäistä keraa. Arvo myös sidotaan muuttujaan, eli jos muuttujaa tarvitaan uudelleen, niin arvoa ei tarvitse evaluoida uudelleen.

By-name-parametri on funktion parametri, joka luodaan funktion määrittelyssä syntaksilla \verb!parameterName: => Type!. Funktiokutsussa parametrin arvo evaluoidaan vasta, kun parametria käytetään. Laiskasta muuttujasta poiketen arvoa ei kuitenkaan evaluointihetkellä sidota parametriin, vaan jos parametria käytetään uudelleen, niin arvokin evaluoidaan uudelleen.

Laiskojen muuttujien ja by-name-parametrien avulla Scalassa pystyy toteuttamaan monia laiskalle evaluoinnille ja puhtaalle funktionaaliselle ohjelmoinnille tyypillisiä piirteitä, kuten omia kontrollirakenteita ja laiskoja listoja \citet{chiusano2014functional}. Siten Scalaa pidetään myös tehokkaana kielenä täsmäkielien toteuttamiseen.

\subsection{Tuki Clojure-ohjelmointikielessä}

Clojure on funktionaalinen, Lisp-kieleä muistuttava yleiskäyttöinen ohjelmointikieli, joka Scalan tapaan kääntyy Java-tavukoodiksi ja on täysin yhteensopiva Java-kirjastojen kanssa. Clojure on semantiikaltaan tiukka kieli, eikä se tue Scalan tapaan yleistetysti laiskoja muuttujia ja parametreja. Kieli kuitenkin tukee laiskoja listoja, joita varten kielessä on valmis tuki \verb!lazy-seq! -makron avulla.

Käytännössä Clojuren käyttäjä tulee käyttäneeksi laiskoja listoja paljon, sillä monet kielen yleisimmistä listaoperaatioista palauttavat laiskan listan. Näitä operaatioita ovat esimerkiksi \verb!map!, \verb!filter! ja \verb!take!. Clojuren laiskojen listojen arvoja evaluoidaan sitä mukaan kuin niitä tarvitaan, ja arvot pysyvät muistissa tulevia käyttökertoja varten.

Clojure tukee \textit{makroja}, joiden avulla kielen kääntäjää pystyy laajentamaan ohjelmakoodista käsin, ja sen myötä kieleen voi luoda omia kontrollirakenteita. Tämän myötä Clojure ei tarvitse laiskaa evaluointia ollakseen silti tehokas kieli täsmäkielien luomiseen.

\subsection{Käyttö JavaScript-apukirjastoissa}

JavaScript on ainoa verkkoselainten yleisesti tukema ohjelmointikieli, ja lisäksi JavaScriptiä käytetään runsaasti myös palvelinohjelmointiin. Se on semantiikaltaan tiukka kieli, ja tukee rajoitetusti funktionaalisen ohjelmoinnin konsepteja.  JavaScriptiin on 2010-luvulla kehitetty useita apukirjastoja, jotka tuovat funktionaalisen ohjelmoinnin työkaluvalikoimaa laajemmin ohjelmoijien käyttöön. Näistä uusimmissa näkyy kasvavissa määrin laiskan evaluoinnin konseptien vaikutus kirjastosuunnitteluun.

\textit{Immutable.js} on Facebookin julkaisema JavaScript-apukirjasto, joka tarjoaa tuen funktionaaliselle ohjelmoinnille ominaisille pysyvätilaisille tietorakenteille. Kaikki kirjastossa määritellyt tietorakenteet, esimerkiksi indeksoidun listan \verb!List! tai pinon \verb!Stack!, voi muuttaa laiskaksi \verb!Seq!-konstruktorilla. Listaoperaatiot ovat tämän luokan metodeja, ja kun \verb!Seq!-instanssille kutsuu näitä metodeja, ne palauttavat uuden laiskan listan.
Ketjutetut listaoperaatiot evaluoidaan vasta sitten, kun lopullista arvoa tarvitaan.

\begin{sloppypar}
\verb!Seq! tukee myös päättymättömiä listoja kirjaston tarjoamien \verb!Range!- ja \verb!Repeat! \mbox{-konstruktoreiden} sekä iteraattorifunktioiden avulla. \verb!Seq! ei kuitenkaan Clojuren laiskojen listojen tapaan pidä evaluoituja arvoja muistissa.
\end{sloppypar}

\textit{Lazy.js} on JavaScriptin-apukirjasto listamaisten tietorakenteiden käsittelemiseen. Siinä Immutablen laiskaa listan luontia vastaa konstruktori \verb!Sequence!. Lazy.js ei Immutablen tapaan esittele uusia tietorakenteita, vaan sitä voi käyttää yhdessä JavaScriptin sisäänrakennettujen tietorakenteiden kanssa. Toisin kuin Immutable.js, se myös tukee joitain funktionaalisen reaktiivisen ohjelmoinnin konsepteja, joita käsittelen seuraavassa luvussa.

Sekä Immutable.js että Lazy.js lupaavat, että viivyttämällä listaoperaatioiden evaluointia ne ovat merkittävästi nopeampia kuin kilpailijansa, joissa tyypillisesti listan arvo evaluoidaan uudelleen jokaisen listaoperaation kutsun yhteydessä.

\subsection{Reaktiivinen ohjelmointi}

Funktionaalinen reaktiivinen ohjelmointi, lyhyemmin FRP, on deklaratiivinen ohjelmointiparadigma, jolla voi käsitellä muuttuvatilaisia arvoja funktionaalisessa ohjelmoinnissa. FRP esittää muuttuvatilaisen arvon aikariippuvaisena ``signaalina'' . Näitä signaaleja voi transformoida ja yhdistellä joustavasti \citep{czaplicki2012elm}.

Paradigma sai alkunsa Haskellin tutkimuksesta, kun \citep{m} 

 \citep{czaplicki2012elm}

\subsection{Apukirjastot}
\begin{itemize}
  \item{Laiska JavaScript-ohjelmointi, Lazy.js jne}
  \item{FRP, Bacon.js, Elm \citep{czaplicki2012elm}}
\end{itemize}
\subsection{Muut}
\begin{itemize}
  \item{Concurrency hässäkät, parallelism?}
  \item{Piirisuunnittelu \citep{charlton1991lazy}}
  \item{Persistentit ohjelmointikielet (aika UG aihe) \citep{wevers2014persistent}}
  \item{Logiikkapohjainen ohjelmointi (jälleen aika UG aihe) \citep{alpuente1997specialization}}
\end{itemize}

% Non-deterministinen ohjelmointi

% "Beyond lazy evaluation": optimistinen evaluointi, lempeä evaluointi

% 9. Nyt valitsen kiinnostavimmat aihealueet/sovellutukset tarkempaan tarkasteluun, ja otan tässä kohtaa myös vahvuus/heikkous -selvityksen mukaan. Perustelen, että laiskaa evaluointia ei ole mieltä tutkia yhtenä kokonaisuutena mielipidearvioinnin keinoin, vaan on järkevämpää keskittyä yksittäisiin sovellutuksiin. Aiheita/sovellutuksia voisi olla esimerkiksi Haskell, Scala, JavaScript-kirjastot ja FRP. Tässä voi saada aikaan mielenkiintoista mielipiteiden, blogien ja tieteellisten artikkeleiden "vuoropuhelua".

\section{Laiskan evaluoinnin edut ja haitat}

It makes it practical to modularise a program as a generator which constructs a large number of possible answers, and a selector which chooses the appropriate one.

%   \item{Lazy I/O \citep[s.123]{o2008real}}

% 10. Mielipiteistä ja sovellutuksista voisi jatkaa vielä kertomalla laiskan evaluoinnin hyödyntämisestä tieteellisessä tutkimuksessa, kuten laskennallisessa biologiassa. Myös yritysesimerkkejä voi ottaa mukaan. Tämä olisi teoriassa yhdistettävissä edelliseen lukuun, mutta toisaalta koen tämän olevan kuitenkin korkeammalla tasolla (kielten ja kirjastojen vahvuudet/heikkoudet vs. laiskan evaluoinnin soveltuvuus yleisemmin eri alojen työhön).

\section{Laiskan evaluoinnin yleisyys tutkimuksessa ja työelämässä}

% 11. Tarinan kaaren kannalta olisi kandin loppupuolella kiva olla tulevaisuuden tarkastelua. Tässä voisin nostaa vielä pari aikaisemmin raportissa käsiteltyä trendiä, jotka vaikuttavat lupaavilta tulevaisuuden osalta, erikseen tarkasteluun. Voin myös koota yhteen, millaisten esteiden raivaaminen voisi auttaa laiskaa evaluointia entistä suurempaan suosioon.

% 12. Yhteenvedon yhteydessä arvioidaan vielä työn metodisia heikkouksia/rajoituksia, ja mitä puutteita niistä on mahdollisesti seurannut. Sitä, mitä muuta yhteenvedossa olisi mielekästä käsitellä, pitää vielä selvittää.

\section{Yhteenveto ja johtopäätökset}

Laiskasta evaluoinnista on kehitetty edelleen uusia evaluointistrategioita. Optimistinen evaluointi, lempeä evaluointi. \citep{ennals2003optimistic} \citep{maessen2002hybrid}

% \subsection{Yhteenveto}
% Työssä jäi vähäiseksi algoritminen ja matemaattinen näkökulma. Miksi se olisi ollut hyödyllistä?

%\clearpage                     % luku loppuu, loput kelluvat tänne, sivunv.

%\input{luku2}                  % tässä tyylissä ei sivunvaihtoja lukujen
%\input{luku3}                  %   välillä. Toiset ohjaajat haluavat
%\input{luku4}                  %   sivunvaihdot.

\label{pages:text}
\clearpage                     % luku loppuu, loput kelluvat tänne, sivunvaihto
%\newpage                       % ellei ylempi tehoa, pakota lähdeluettelo
                               % alkamaan uudelta sivulta

% -------------- Lähdeluettelo / reference list -----------------------
%
% Lähdeluettelo alkaa aina omalta sivultaan; pakota lähteet alkamaan
% joko \clearpage tai \newpage
%
%
% Muista, että saat kirjallisuusluettelon vasta
%  kun olet kääntänyt ja kaulinnut "latex, bibtex, latex, latex"
%  (ellet käytä Makefilea ja "make")

% Viitetyylitiedosto aaltosci_t.bst; muokattu HY:n tktl-tyylistä.
\bibliographystyle{aaltosci_t}
% Katso myös tämän tiedoston yläosan "preamble" ja siellä \bibpunct.

% Muutetaan otsikko "Kirjallisuutta" -> "Lähteet"
\renewcommand{\refname}{Lähteet}  % article-tyyppisen
%\renewcommand{\bibname}{Lähteet}  % jos olisi book, report-tyyppinen

% Lisätään sisällysluetteloon
\addcontentsline{toc}{section}{\refname}  % article
%\addcontentsline{toc}{chapter}{\bibname}  % book, report

% Määritä kaikki bib-tiedostot
\bibliography{sources}
%\bibliography{thesis_sources,ietf_sources}

\label{pages:refs}
\clearpage         % erotetaan mahd. liitteet alkamaan uudelta sivulta

% -------------- Liitteet / Appendices --------------------------------
%
% Liitteitä ei yleensä tarvita. Kommentoi tällöin seuraavat
% rivit.

% Tiivistelmässä joskus matemaattisen kaavan tarkempi johtaminen,
% haastattelurunko, kyselypohja, ylimääräisiä kuvia, lyhyitä
% ohjelmakoodeja tai datatiedostoja.

\appendix
%\section{Esimerkkiliite}
\label{sec:app1}

Jos työhön kuuluu suurikokoisia (yli puoli sivua) kuvia, taulukoita
tai karttoja tms., jotka eivät kokonsa puolesta sovi tekstin joukkoon,
ne laitetaan liitteisiin. Liitteet numeroidaan. Jokaiseen liitteeseen
tulee viitata tekstissä, eikä liitteisiin ole tarkoitus laittaa ``mitä
tahansa'', vaan vain työlle oikeasti tarpeellista
materiaalia. Liitteisiin voidaan sijoittaa esim. malli
kyselylomakkeesta, jolla tutkimushaastattelu toteutettiin,
pohjapiirustuksia, taulukoita, kaavioita, kuvia tms.

\textbf{TIK.kand suositus: Vältä liitteitä.} Jos iso kuva, mieti onko
sen koko pienettävissä (täytyy olla tulkittavissa) normaalin tekstin
yhteyteen. Joskus liitteeksi lisätään matemaattisen kaavan tarkempi
johtaminen, haastattelurunko, kyselypohja, ylimääräisiä kuvia, lyhyitä
ohjelmakoodeja tai datatiedostoja.

Työtä varten mahdollisesti tehtyjä ohjelmakoodeja ei tyypillisesti
lisätä tänne, ellei siihen ole joku erityinen syy. (Kukaan ei ala
kirjoittaa tai tarkistamaan koko koodia paperilta vaan pyytää sitä
sinulta, jos on kiinnostunut.)

%\subsection{Esimerkkiliitteen otsikko 1}
%\label{sec:app1_1}
%
%Kerätty data-aineisto.
%
% -------------------------------------------------------------- %
%
%\newpage
%\section{Toinen esimerkkiliite}
%\label{sec:app2}
%
%Haastattelukysymykset: mitä, missä, milloin, kuka, miten.



\label{pages:appendices}

% ---------------------------------------------------------------------

\end{document}
