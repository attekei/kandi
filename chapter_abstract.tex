% --------------------------------------------------------------------
% Cheatsheet:
% \verb!lazy val!   -- esim. lyhyisiin koodinpätkiin
% \begin{verbatim}  -- esim. pitkiin koodinpätkiin
% `` tekstiä ''     -- lainausmerkit
% \textbf{huom!}    -- boldaus
% \begin{quotation}
% \noindent \it     -- quoten lisäys
% \ldots            -- kolme pistettä
% footnote{juh}     -- alaviite
% \citet, \citep
% \citet[s.234]{}   -- viitteet, http://merkel.zoneo.net/Latex/natbib.php
% \begin{sloppypar} -- ahdas teksti
% \clearpage        -- onkkelmien korjaamiseen lukujen kanssa
% --                -- yhdysviiva
% \enumerate
% \itemize          -- listat
% \begin[htb]{figure}
% \begin[htb]{table} - Kuva ja taulukko, (h)ere, (t)op, (b)ottom
% \begin{equation}  -- kaava
% \label{eq:kaava1} -- laabeli
%!TEX root = main.tex

% Tiivistelmät tehdään viimeiseksi.
%
% Tiivistelmä kirjoitetaan käytetyllä kielellä (JOKO suomi TAI ruotsi)
% ja HALUTESSASI myös samansisältöisenä englanniksi.
%
% Avainsanojen lista pitää merkitä main.tex-tiedoston kohtaan \KEYWORDS.

\begin{fiabstract}
\textit{Kirjoitan tiivistelmän vasta viimeisenä.}
% Tiivistelmän tyypillinen rakenne:
% (1) aihe, tavoite ja rajaus
% (heti alkuun, selkeästi ja napakasti, ei johdattelua);
% (2) aineisto ja menetelmät (erittäin lyhyesti);
% (3) tulokset (tälle enemmän painoarvoa);
% (4) johtopäätökset (tälle enemmän painoarvoa).

%
%Tiivistelmätekstiä tähän (\languagename). Huomaa, että tiivistelmä tehdään %vasta kun koko työ on muuten kirjoitettu.
\end{fiabstract}

%\begin{svabstract}
%  Ett abstrakt hit
%%(\languagename)
%\end{svabstract}

%\begin{enabstract}
% Here goes the abstract
%%(\languagename)
%\end{enabstract}
