% --------------------------------------------------------------------
% Cheatsheet:
% \verb!lazy val!   -- esim. lyhyisiin koodinpätkiin
% \begin{verbatim}  -- esim. pitkiin koodinpätkiin
% `` tekstiä ''     -- lainausmerkit
% \textbf{huom!}    -- boldaus
% \begin{quotation}
% \noindent \it     -- quoten lisäys
% \ldots            -- kolme pistettä
% footnote{juh}     -- alaviite
% \citet, \citep
% \citet[s.234]{}   -- viitteet, http://merkel.zoneo.net/Latex/natbib.php
% \begin{sloppypar} -- ahdas teksti
% \clearpage        -- onkkelmien korjaamiseen lukujen kanssa
% --                -- yhdysviiva
% \enumerate
% \itemize          -- listat
% \begin[htb]{figure}
% \begin[htb]{table} - Kuva ja taulukko, (h)ere, (t)op, (b)ottom
% \begin{equation}  -- kaava
% \label{eq:kaava1} -- laabeli
%!TEX root = main.tex

\section{Yhteenveto ja johtopäätökset}\label{yhteenveto}

% Laiskasta evaluoinnista on kehitetty edelleen uusia evaluointistrategioita. Optimistinen evaluointi, lempeä evaluointi. \citep{ennals2003optimistic} \citep{maessen2002hybrid}

\textit{Haluan suoda ajatuksia tuloksista vedettäville johtopäätöksille vasta sitten, kun tuloksia käsittelevät luvut ovat jokseenkin lopullisessa muodossa. Minkähäntyyppisiä johtopäätöksiä voin ylipäätään vetää? Mikä on yhteenvedon rooli?}

Metodologisia puutteita:

- varsin rajallinen määrä mielipiteitä mielipidetutkimuksessa

- kandin pituus on melko lyhyt, ja siksi kirjallisuuskatsaus jää väistämättä pintapuoliseksi




% \subsection{Yhteenveto}
% Työssä jäi vähäiseksi algoritminen ja matemaattinen näkökulma. Miksi se olisi ollut hyödyllistä?
