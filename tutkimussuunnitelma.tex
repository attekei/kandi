\documentclass[12pt,a4paper,finnish,oneside]{article}

% Valitse 'input encoding':
%\usepackage[latin1]{inputenc} % merkistökoodaus, jos ISO-LATIN-1:tä.
\usepackage[utf8]{inputenc}   % merkistökoodaus, jos käytetään UTF8:a
% Valitse 'output/font encoding':
%\usepackage[T1]{fontenc}      % korjaa ääkkösten tavutusta, bittikarttana
\usepackage{ae,aecompl}       % ed. lis. vektorigrafiikkana bittikartan sijasta
% Kieli- ja tavutuspaketit:
\usepackage[finnish]{babel}
% Muita paketteja:
% \usepackage{amsmath}   % matematiikkaa
\usepackage{url}       % \url{...}

% Kappaleiden erottaminen ja sisennys
\parskip 1ex
\parindent 0pt
\evensidemargin 0mm
\oddsidemargin 0mm
\textwidth 159.2mm
\topmargin 0mm
\headheight 0mm
\headsep 0mm
\textheight 246.2mm

\pagestyle{plain}

% ---------------------------------------------------------------------

\begin{document}

% Otsikkotiedot: muokkaa tähän omat tietosi

\title{TIK.kand tutkimussuunnitelma:\\[5mm] Laiska evaluointi funktionaalisessa ohjelmoinnissa}

\author{Atte Keinänen\\
Aalto-yliopisto\\
\url{atte-keinanen@aalto.fi}}

\date{\today}

\maketitle

% MUOKKAA TÄHÄN. Jos tarvitset tähän viitteitä, käytä
% tässä dokumentissa numeroviitejärjestelmää komennolla \cite{kahva}.
%
% Paljon kandidaatintöitä ohjanneen Vesa Hirvisalon tarjoama
% sabluuna. Kursivoidut osat \emph{...} ovat kurssin henkilökunnan
% lisäämiä.

\textbf{Kandidaatintyön nimi:} Laiska evaluointi funktionaalisessa ohjelmoinnissa

\textbf{Työn tekijä:} Atte Keinänen, Informaatioverkostot

\textbf{Ohjaaja:} Juha Sorva, Tietotekniikan laitos


\section{Tiivistelmä tutkimuksesta}

Ohjelmointikielissä on vaihtelevia evaluointistrategioita, eli käytäntöjä sille, milloin ja miten lausekkeen arvo evaluoidaan. Valtavirtaa ohjelmoinnissa on ahne evaluointi, eli lauseke evaluoidaan välittömästi, kun lauseke sijoitetaan muuttujaan.

Kuitenkin suosioon on noussut Haskellin, OCamlin ja Scalan kaltaisia ohjelmointikieliä, jotka tukevat myös laiskaa evaluointia. Tällä tarkoitetaan sitä, että lausekkeen arvo evaluoidaan vasta, kun sen arvoa tarvitaan. Laiskalla evaluoinnilla pystytään välttämään tarpeettomia lausekkeiden evaluointeja, ja toisaalta se mahdollista päättymättömän listan kaltaisia tietorakenteita. Toisaalta laiskaa evaluointia kritisoidaan siitä, että sitä käyttävien ohjelmien muistinkulutusta ja koodin suoritusjärjestystä on hankalaa ennakoida.

Kandityössä tarkastellaan laiskan evaluoinnin historiaa ja vaikutuksia tietojenkäsittelytieteen sekä ohjelmointikielten kehitykseen. Työssä tarkastellaan myös sitä, miten laiska evaluointi on moderneissa ohjelmointikielissä toteutettu, ja mihin sitä tänä päivänä käytetään niin akateemisessa tutkimuksessa kuin yritysmaailmassa.

\section{Tavoitteet ja näkökulmat}

Juuh jotain tänne

Mitä haluat saada selville? Mitkä ovat keskeisiä kysymyksiä? Mistä
näkökulmasta asiaa tarkastellaan?

Tutkimuskysymystä kannattaa siis rajata ja tarkentaa sekä huomioida
näkökulman merkitys. Ts. jänikset eläintieteen kannalta ovat eri aihe
kuin jänikset metsästyksen näkökulmasta.

\section{Tutkimusmateriaali}

Millaisen aineiston varaan perustat tutkimuksesi? Arvioi materiaalin
riittävyyttä asetettuihin tavoitteisiin nähden.

Pitää olla siis hieman kuvaa siitä, minkälaisen materiaalin kanssa
ollaan tekemisissä ja mitä sellaisen käsittelyyn tarvitaan (etenkin
siis tarvittavan ajan puolesta; ts. kuinka monta tuntia/minuuttia per
lähde?).

\section{Tutkimusmenetelmät}

Miten sen keräät materiaalisi tai saat sen käsiisi? Kuinka käsittelet
sen? Kuinka siitä tulee raportti?

Tavallaisesti kirjallisuustutkimuksen yhteydessä tämä on:
(a) lähderyhmien valinta,
(b) viitteiden ja lähteiden haku,
(c) lähteiden arviointi,
(d) lähteiden lukeminen,
(e) tiedon organisointi,
(f) raportointi.  % (f) tärkeää ettei jää vain lukemiseksi!

Kirjallisuustutkimuksen yleinen menetelmä pitää sovittaa tähän
nimenomaiseen aiheeseen sekä tekijän lähtökohtiin. Kuinka sinä teet
muistiinpanot (että myös kirjoitat etkä pelkästään lue). Eli tälle
pitää hieman miettiä omakohtaista vaiheistusta. Siis nähdä ihan
oikeasti, kuinka sinä saat tutkielman tehtyä.

Ja... raportointi ei ole kirjoittamista vaan jo kirjoitettujen
muistiinpanojen koostamista yhteneväksi teokseksi.

\section{Haasteet}

Yleensä kaikkiin töihin liittyy kompastuskiviä. Ne on syytä tiedostaa
etukäteen. Yhdessä työssä aihe on suurpiirteinen (työn rajaaminen
vaikeaa), toisessa materiaalia on niukasti saatavissa, kolmannessa
taas materiaalia on hukkumiseen asti.  Eli, nämä pitäisi kyseisen
tutkimuksen osalta kirjata ylös, ja nähdä ne myös mahdollisuuksina
(positiivisina haasteina) ei ainostaan esteinä.

\section{Resurssit}

Kuka tätä työtä tekee, kuka ohjaa, jne. Paljonko on käyttää
aikaa. Tarvitaanko muuta? (Onko työssä joku kokeellinen osuus?)

\section{Aikataulu}

Laadi tutkimustyölle ja raportoinnille realistinen aikataulu.
Huolehdi, että suunnitelmasi vastaa kandidaatin tutkielman sekä
seminaarin aikataulua sekä laajuutta.  \emph{Kurssiesitteessä omalle
  kirjoitusprosessille on arvioitu noin 6 op eli 160 tuntia eli noin 4
  viikkoa työtä.}

%\begin{tabular}{|p{30mm}|p{120mm}|}
%\hline
%abcd   & qwerty qwerty \\ \hline
%abcd   & qwerty qwerty \\ \hline
%abcd   & qwerty qwerty \\ \hline
%\end{tabular}


\section{Esittäminen}

Laadi lyhyt sisällysluettelo, jossa on hahmoteltuna kandityön pää- ja
alaluvut. Yleensä perusrunko on
(1) Johdanto,
(2) Tausta,
(3) Sisäluvut,
(4) Yhteenveto.
%
Sinun täytyy suunnitella oma raportointisi tähän sopivaksi.

\emph{Rakenne tarkentuu työn edetessä. Tutkimussuunnitelmaan ei välttämättä tarvita lähdeluetteloa, mutta halutessasi voit sisällyttää tärkeimmät lähteet.}

% ---------------------------------------------------------------------
%
% ÄLÄ MUUTA MITÄÄN TÄÄLTÄ LOPUSTA

% Tässä on käytetty siis numeroviittausjärjestelmää.
% Toinen hyvin yleinen malli on nimi-vuosi-viittaus.

% \bibliographystyle{plainnat}
\bibliographystyle{finplain}  % suomi
%\bibliographystyle{plain}    % englanti
% Lisää mm. http://amath.colorado.edu/documentation/LaTeX/reference/faq/bibstyles.pdf

% Muutetaan otsikko "Kirjallisuutta" -> "Lähteet"
\renewcommand{\refname}{Lähteet}  % article-tyyppisen

% Määritä bib-tiedoston nimi tähän (eli lahteet.bib ilman bib)
\bibliography{lahteet}

% ---------------------------------------------------------------------

\end{document}
