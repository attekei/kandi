\documentclass[12pt,a4paper,finnish,oneside]{article}

% Valitse 'input encoding':
%\usepackage[latin1]{inputenc} % merkistökoodaus, jos ISO-LATIN-1:tä.
\usepackage[utf8]{inputenc}   % merkistökoodaus, jos käytetään UTF8:a
% Valitse 'output/font encoding':
%\usepackage[T1]{fontenc}      % korjaa ääkkösten tavutusta, bittikarttana
\usepackage{ae,aecompl}       % ed. lis. vektorigrafiikkana bittikartan sijasta
% Kieli- ja tavutuspaketit:
\usepackage[finnish]{babel}
% Muita paketteja:
% \usepackage{amsmath}   % matematiikkaa
\usepackage{url}       % \url{...}

% Kappaleiden erottaminen ja sisennys
\parskip 1ex
\parindent 0pt
\evensidemargin 0mm
\oddsidemargin 0mm
\textwidth 159.2mm
\topmargin 0mm
\headheight 0mm
\headsep 0mm
\textheight 246.2mm

\pagestyle{plain}

% ---------------------------------------------------------------------

\begin{document}

% Otsikkotiedot: muokkaa tähän omat tietosi

\title{TIK.kand tutkimussuunnitelma:\\[5mm] Laiska evaluointi funktionaalisessa ohjelmoinnissa}

\author{Atte Keinänen\\
Aalto-yliopisto\\
\url{atte-keinanen@aalto.fi}}

\date{\today}

\maketitle

% MUOKKAA TÄHÄN. Jos tarvitset tähän viitteitä, käytä
% tässä dokumentissa numeroviitejärjestelmää komennolla \cite{kahva}.
%
% Paljon kandidaatintöitä ohjanneen Vesa Hirvisalon tarjoama
% sabluuna. Kursivoidut osat \emph{...} ovat kurssin henkilökunnan
% lisäämiä.

\textbf{Kandidaatintyön nimi:} Laiska evaluointi funktionaalisessa ohjelmoinnissa

\textbf{Työn tekijä:} Atte Keinänen, Informaatioverkostot

\textbf{Ohjaaja:} Juha Sorva, Tietotekniikan laitos


\section{Tiivistelmä tutkimuksesta}

Ohjelmointikielissä on vaihtelevia evaluointistrategioita, eli käytäntöjä sille, milloin ja miten lausekkeen arvo evaluoidaan. Valtavirtaa ohjelmoinnissa on ahne evaluointi, eli lauseke evaluoidaan välittömästi, kun lauseke sijoitetaan muuttujaan.

Kuitenkin suosioon on noussut Haskellin, OCamlin ja Scalan kaltaisia ohjelmointikieliä, jotka tukevat myös laiskaa evaluointia. Tällä tarkoitetaan sitä, että lausekkeen arvo evaluoidaan vasta, kun sen arvoa tarvitaan. Laiskalla evaluoinnilla pystytään välttämään tarpeettomia lausekkeiden evaluointeja, ja toisaalta se mahdollista päättymättömän listan kaltaisia tietorakenteita. Toisaalta laiskaa evaluointia kritisoidaan siitä, että sitä käyttävien ohjelmien muistinkulutusta ja koodin suoritusjärjestystä on hankalaa ennakoida.

Kandityössä tarkastellaan laiskan evaluoinnin historiaa ja vaikutuksia tietojenkäsittelytieteen sekä ohjelmointikielten kehitykseen. Työssä tarkastellaan myös sitä, miten laiska evaluointi on moderneissa ohjelmointikielissä toteutettu, ja mihin sitä tänä päivänä käytetään niin akateemisessa tutkimuksessa kuin yritysmaailmassa.

\section{Tavoitteet ja näkökulmat}

Halusin valita itse kandiaiheeni, koska puhdas funktionaalinen ja erityisesti Haskell kiinnostivat minua. En ollut ennestään käyttänyt puhtaasti funktionaalisia ohjelmointikieliä, ja olin työpaikkani kautta kuullut, että Haskelliin tutustuminen olisi hyvä tapa oppia ymmärtämään funktionaalista ohjelmointia ja sen kehitystä syvemmin. Ennen aiheen valintaa luin kirjan Learn You a Haskell for Great Good \cite{lipovaca2011learn} ja Haskellin tyyppijärjestelmää käsittelevän Typeclassopedian \cite{yorgey2009typeclassopedia}. Nämä auttoivat löytämään kiinnostavia aihepiirejä Haskellin liittyen.

Päädyin rajaamaan aiheen laiskaan evaluointiin. Laiska evaluointi on monille puhtaasti funktionaalisille ohjelmointikielille yhteistä.

\emph{Onkohän tämä luku tähänastisen tekstin osalta miltei turhankin taustoittava? Katsotaan joka tapauksessa tähän aiheen tarkempi rajaus ja työn tavoitteet torstaina.}

\section{Tutkimusmateriaali}

\begin{sloppypar}
Laiskasta evaluoinnista on runsaasti akateemisia artikkeleita. Esimerkiksi Haskell -ohjelmointikielen kehityksen taustalla oli useita jo klassikonomaiseen asemaan nousseita artikkeleita, joissa perusteltiin laiskan evaluoinnin hyötyjä \cite{hudak2007history}. Samoin erilaisista spesifeistä ongelmista esimerkiksi tekoälytutkimuksen saralta, joiden käytössä laiska evaluointi on osoittautunut hyödylliseksi, on paljon saatavilla.
\end{sloppypar}

Akateemisten artikkeleiden lisäksi luontevaa on seurata funktionaalisen ohjelmoinnin piirissä arvostettujen ihmisten blogeja ja kirjoituksia. Näiden avulla saadaan vastattua pelkkiä akateemisia tutkimuksia kattavammin siihen, mikä on käsitys laiskan evaluoinnin suosiosta ja käyttökohteista tällä hetkellä.

Aikaa yksittäisen artikkelin menee luultavasti noin 30 minuuttia per artikkeli. Koska kyseessä on melko tyylipuhdas kirjallisuustutkimus, artikkeleita tarvitsee kerätä arviolta 20-30. Lukemisen lisäksi aikaa menee kerätyn tiedon järjestämiseen, mistä lisää seuraavassa luvussa.

\section{Tutkimusmenetelmät}

Kerätyt artikkelit kootaan Trello-järjestelmään aihepiireittäin siten, että kullekin artikkelille luodaan oma Trello-korttinsa, josta on linkki alkuperäiseen artikkeliin. Yksittäinen artikkeli voi kuulua useampaan aihepiiriin. Aihepiirijako on alkuvaiheessa seuraavanlainen, ja sitä muutetaan ja kasvatetaan lähdemateriaalin määrän kasvaessa:

\begin{itemize}
  \item Varhaishistoria
  \item Vaikutukset tietojenkäsittelytieteeseen
  \item Toteutus nykyaikaisissa ohjelmointikielissä
  \item Akateeminen käyttö tänä päivänä
  \item Yrityskäyttö tänä päivänä
\end{itemize}

Kustakin artikkelista eritellään Trellossa kommenttitoiminnon avulla poimintoja ja muistiinpanoja aihepiireittäin. Todennäköisesti päädyn käyttämään yhdistelmää suoria lainauksia artikkeleista että itse kirjoittamiani lyhyitä tiivistelmiä.

Huomionarvoista on myös, että aina kun lisään artikkelin Trelloon, lisään sen myös BibTeX-lähdeluetteloon. Näin varsinaista kandidaatintyötä kirjoittaessani minun ei tarvitse enää erikseen etsiä lähdeviittauksia.

Aineiston keruun Trello-taulu on avoimesti nähtävillä seuraavassa osoitteessa:

\url{https://trello.com/b/Q9ZenGG4}

\emph{Vielä lisätarkennuksia tarvitsee, etenkin siihen miten raportin tekeminen tapahtuu. Käsiteltäviä asioita (osa näistä onkin jo käsitelty):
(a) lähderyhmien valinta,
(b) viitteiden ja lähteiden haku,
(c) lähteiden arviointi,
(d) lähteiden lukeminen,
(e) tiedon organisointi,
(f) raportointi. }

\section{Haasteet}

Työn haasteet liittyvät toisiinsa liittyen työn rajaukseen ja siihen, että materiaalia on huomattavan paljon saatavilla.

Laiskan evaluaation sisältä pystyisi kandin melko suppean laajuuden huomioiden käsittelemään halutessaan vain yksittäistäkin osa-aluetta, esimerkiksi sen hyödyntämistä yritysmaailmassa, tai teknistä toteutusta parissa eri ohjelmointikielessä. Tämä on hyvä pitää mielessä, jos vaikuttaa siltä, jos kandin sivumäärä meinaa karata käsistä.

Tehdyn tutkimuksen suuri määrä johtaa ennen kaikkea siihen, että tarjolla olevaa artikkelivalikoimaa on hyvä arvioida ennen artikkeleihin syventymistä, ja näin valita tieteellisesti vakuuttavimmat ja oman aiheen kannalta relevanteimmat artikkelit. Trelloa voisi luontevasti käyttää siihen, että ensiksi koostaa laajemman listan artikkeleista, jota tarvittaessa karsii, ja sitten vasta lukee artikkelit syvällisemmin ja tekee niistä muistiinpanot.

\section{Resurssit}

Työssä ei ole muita osapuolia kuin minä ja työn ohjaaja Juha Sorva. Juhan kanssa olemme tähän asti saaneet hyvin sujuvasti sovittua yksittäiset tapaamiset, ja sen kanssa tuskin tulee jatkossakaan ongelmia. Tapaamme hieman kurssin virallista tapaamissykliä tiheämmin, koska suoritan kurssin tavallisesta poikkealla aikataululla.

\emph{Vähän keinotekoinen luku, voisi luontevasti yhdistää aikatauluun.}

\section{Aikataulu}

Kandityöstä on tavoitteenani saada V3, eli viimeistelemätön mutta täyspitkä versio, valmiiksi jo 24. helmikuuta mennessä. Syynä on, että lähden 25. helmikuuta vaihto-opiskelemaan Etelä-Koreaan, enkä halua jättää työtä täysin keskeneräiseksi siinä kohtaa.

Yhdistän kandin suorituksen kesäkandikurssiin. Palattuani Etelä-Koreasta heinäkuun lopussa jatkan kandityön viimeistelyn parissa, ja osallistun elokuussa niille luennoille, joihin en nyt keväällä. Hoidan seminaarin, opponoinnin ja kypsyysnäytteen kesäkurssin aikataulusta riippuen elokuun lopulla tai syyskuun alussa.

\emph{Tähän vielä tarkempi aikataulu, josta voidaan torstaina jutella. Voisi vaikka käyttää taulukkoformaattia hyödyksi.}
%\begin{tabular}{|p{30mm}|p{120mm}|}
%\hline
%abcd   & qwerty qwerty \\ \hline
%abcd   & qwerty qwerty \\ \hline
%abcd   & qwerty qwerty \\ \hline
%\end{tabular}

\section{Esittäminen}

\emph{Katsotaan tätä torstaina. Tarvitsen ideoita hyvään sisällysluettelorakenteeseen, joka tukisi tiedonhaun tyyliäni hyvin.}

% ---------------------------------------------------------------------
%
% ÄLÄ MUUTA MITÄÄN TÄÄLTÄ LOPUSTA

% Tässä on käytetty siis numeroviittausjärjestelmää.
% Toinen hyvin yleinen malli on nimi-vuosi-viittaus.

% \bibliographystyle{plainnat}
\bibliographystyle{finplain}  % suomi
%\bibliographystyle{plain}    % englanti
% Lisää mm. http://amath.colorado.edu/documentation/LaTeX/reference/faq/bibstyles.pdf

% Muutetaan otsikko "Kirjallisuutta" -> "Lähteet"
\renewcommand{\refname}{Lähteet}  % article-tyyppisen

% Määritä bib-tiedoston nimi tähän (eli lahteet.bib ilman bib)
\bibliography{lahteet}

% ---------------------------------------------------------------------

\end{document}
