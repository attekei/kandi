% --------------------------------------------------------------------
% Cheatsheet:
% \verb!lazy val!   -- esim. lyhyisiin koodinpätkiin
% \begin{verbatim}  -- esim. pitkiin koodinpätkiin
% `` tekstiä ''     -- lainausmerkit
% \textbf{huom!}    -- boldaus
% \begin{quotation}
% \noindent \it     -- quoten lisäys
% \ldots            -- kolme pistettä
% footnote{juh}     -- alaviite
% \citet, \citep
% \citet[s.234]{}   -- viitteet, http://merkel.zoneo.net/Latex/natbib.php
% \begin{sloppypar} -- ahdas teksti
% \clearpage        -- onkkelmien korjaamiseen lukujen kanssa
% --                -- yhdysviiva
% \enumerate
% \itemize          -- listat
% \begin[htb]{figure}
% \begin[htb]{table} - Kuva ja taulukko, (h)ere, (t)op, (b)ottom
% \begin{equation}  -- kaava
% \label{eq:kaava1} -- laabeli
%!TEX root = main.tex

% 9. Nyt valitsen kiinnostavimmat aihealueet/sovellutukset tarkempaan tarkasteluun, ja otan tässä kohtaa myös vahvuus/heikkous -selvityksen mukaan. Perustelen, että laiskaa evaluointia ei ole mieltä tutkia yhtenä kokonaisuutena mielipidearvioinnin keinoin, vaan on järkevämpää keskittyä yksittäisiin sovellutuksiin. Aiheita/sovellutuksia voisi olla esimerkiksi Haskell, Scala, JavaScript-kirjastot ja FRP. Tässä voi saada aikaan mielenkiintoista mielipiteiden, blogien ja tieteellisten artikkeleiden "vuoropuhelua".

\section{Subjektiiviset edut ja haitat} \label{subjektiiviset-edut}

\textit{Tulen kirjoittamaan tähän etujen/haittojen kartoitusprosessista sekä siitä, millaisiin alalukuihin olen jakanut osion ja miksi. Voi myös vähän käsitellä sitä miksi Haskell dominoi keskustelua. Se vie laiskan evaluoinnin selvästi pisimmälle suosituista kielistä, ja siten sitä on hyvä käyttää yleisenä benchmarkina laiskaan evaluointiin liittyvässä keskustelussa, toki terveen kritiikin kera. }

\subsection{Ilmaisuvoima}
\citet{hughes1989functional} sanoo, että laiskan evaluoinnin käyttö tekee käytännölliseksi jakaa sovelluskoodia \textit{tuottajiin} ja \textit{valitsimiin}. Tuottajat määrittävät suuren määrän mahdollisia vastauksia, ja valitsin valitsee niistä relevanteimmat. Tämä tarjoaa sellaisen tavan modularisoida sovelluskoodia, joka on laiskasti evaluoiduille puhtaasti funktionaalisille kielille uniikki. Hughes kuvaa, että kyseessä on kenties tehokkain tapa modularisoida koodia funktionaalisen ohjelmoijan työpakissa.

\citet{vesakarvonen} puolestaan väittää, että laiskuus ei merkittävästi kasvata kielen ilmaisuvoimaa. Useat sellaiset kirjastot, joiden väitetään olevan toteutettavissa vain laiskasti evaluoidun kielen avulla, ovat hänen mukaansa helposti toteutettavissa ei-laiskalla kielellä simulomalla laiskuutta muilla keinoin niissä kohti kirjastoja, kun sille on tarve.

Päättymättömät listat \citet{pointoflaziness} toteaa tietorakenteeksi, jonka toteuttamiseksi laiska evaluointi ei ole välttämätön. Hänen mukaansa laiskuudella ja päättymättömien listojen tyyppisillä tietorakenteilla ei ole selkeää yhteyttä toisiinsa. ML-ohjelmointikielessä, joka on tiukasti evaluoitu, samat saman konseptin toteuttaminen onnistuu myös vaivatta.

Sekä Karvonen että \citet{pointoflaziness} sanovat, että laiskuuden tähden Haskellista puuttuu tuki induktiivisille tietotyypeille, joka löytyy esimerkiksi kilpailevasta ML-ohjelmointikielestä. Tämän myötä kielellä ei ole mahdollista esimerkiksi luoda tietotyyppiä luonnollisten lukujen kuvaamiseen, tai tyyppi luonnosllisten lukujen muodostamalle listalle.

\subsection{Ajan- ja tilankäytön ennustettavuus}

Ajankäytön ennustettavuudella tarkoitetaan sitä, kuinka tarkkaan ohjelmoija pystyy arvioimaan, miten ajallisesti ohjelman suoritus etenee, ja kuinka paljon aikaa mikäkin operaatio vie. Tilankäytön ennustettavuus kuvaa vastaavasti sitä, kuinka tarkasti ohjelman muistinkäytön pystyy arvioimaan ohjelman suorituksen edetessä.

Haskell-ohjelmaa tehdessään \citet{sampson2009experience} kohtasi tiiminsä kanssa ongelmia yllättävien tilavuotojen muodossa. Tilavuodot johtuivat siitä, että kielen evaluoimattomat lausekkeet pitävät jostakin syystä sovelluksen näkökulmasta jo vanhentunutta tietoa muistissa. Syiden hän epäili liittyvän koodin rinnakkaisuuden toteutuksessa käytettyihin abstraktioihin.

\citet{vesakarvonen} sanoo, että funktioiden yhdistämiseen liittyvät mahdollisuudet eivät ole Haskellissa parempia, toisin kuin usein väitetään. Esimerkiksi listaoperaatioiden yhdistämisestä ei tule käytännön hyötyä siksi, että on vaikeaa ennakoida, kuinka listaoperaatiot tarkalleen käyttäytyvät laiskasti evaluoidussa kielessä. Tämän seurauksena niiden tilankäytön ennustettavuuskin on heikko.

Artikkelissa X [pitää vielä kaivaa, mikä oli kyseessä] todetaan, että Haskell soveltuu heikosti reaaliaikaiseen grafiikkaan, esimerkiksi peleihin, koska evaluoimattomien lausekkeiden evaluointi isona kertaoperaationa saattaa pysäyttää grafiikan ajaksi, jonka käyttäjä huomaa häiritsevänä nykimisenä.

\subsection{Vianetsintä ja suoritusjärjestys}

Toistuvasti esiin noussut kritiikki laiskaa evaluointia ja Haskellia kohtaan on, että koodin ajonaikaisten ongelmien selvittäminen on hankalaa. Näin kokee muun muassa \citet{daniels2012experience}, jotka ilmaisivat syyksi, että testausta paljon helpottavalle \textit{funktiokutsupinojen seuraamiselle} on laiskoissa funktionaalisissa kielissä heikko tuki. He toisaalta viittasivat myös siihen, että Haskellin kehittyessä kutsupinojen seuraaminen saattaa helpottua.

Samaan tapaan \citet{pop2010experience} kokevat ajonaikaisten ongelmien löytämisen vaikeaksi funktiokutsupinojen seurannan puutteen vuoksi. Haskellin tarjoamat vaihtoehdot ongelmien etsimiseen olivat hyvin primitiivisiä, ja yhdessä laiskuuden kanssa se teki hänen mielestään ongelmien löytämisestään aloittelevalle Haskell-kehittäjälle paljon vaikeampaa verrattuna perinteisempään skriptikieleen.

\citet{sampson2009experience} kokee, että oli vaikeaa hahmottaa, missä tilanteessa mikäkin arvo evaluoidaan, eli milloin ohjelma tekee työtä ja milloin ei. Apukirjastot usein palauttavat evaluoimattomia arvoja, jotka muodostuvat isosta määrästä lausekkeita, ja niiden evaluointi tapahtuu vasta sitten, kun arvoja käytetään. Arvon käyttämisen kohtaa voi olla kuitenkaan kirjoittajien mielestä vaikeaa löytää. Tämä liittyy hänen mukaansa myös ajonaikaiseten virheiden paikantamisen vaikeuteen.

\subsection{Käytön laajuus}
\citet{vesakarvonen} sanoo laiskan evaluoinnin olevan ominaisuus, josta on tietyissä tilanteissa hyötyä, mutta ne kannattaa rajata tarkkaa. Siten hän kannattaa tiukasti evaluoituja ohjelmointikieliä ja laiskan evaluoinnin konseptien käyttöä niissä harkitusti.

\subsection{Koettu arvo}

\citet{sampson2009experience} toteaa artikkelin lopussa, että hän ja tiimi ovat projektin jälkeen epävarmoja laiskan evaluoinnin tuomasta arvosta: joka kerta kun he tuntevat saaneensa siitä hyvän otteen uusia ongelmia ilmenee. Hän kokee, että aiheesta pitäisi vähintäänkin olla olemassa hyvä kirja, joka käsittelisi kaikkia laiskan evaluoinnin kanssa ilmeneviä ongelmia sekä auttaisi luomaan lukijalle hyvä intuitio laiskasti evaluoitujen systeemien käyttäytymisestä.
