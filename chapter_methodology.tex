% DONE
\section{Tutkimusmenetelmät}\label{metodologia}

Tutkimuskysymykseen 1 vastaan hakemalla yleisimpiä käsitteitä, joita laiskaa evaluointia käsittelevässä kirjallisuudessa esiintyy, hakemalla näille yleisimmät määritelmät, ja vertailemalla käsitteiden vaihtoehtoisia määritelmiä. Jäsentelen myös käsitteiden suhteita toisiinsa.

Tutkimuskysymyksessä 2 laiskan evaluoinnin merkityksen tutkiminen tarkoittaa, että tarkastelen (a) laiskan evaluoinnin varhaista historiaa pohjustuksenomaisesti, (b) laiskan evaluoinnin leviämistä moderneihin ohjelmointikieliin, (c) laiskan evaluoinnin kautta kehittyneitä sovellutuksia ohjelmointikielissä ja (d) laiskaa evaluointia käyttävien teknologioiden hyödyntämistä tutkimuksessa ja työelämässä.

Tutkimuskysymyksessä 3 selvitän sitä, millaisia subjektiivisia mielipiteitä laiskaa evaluointia hyödyntäviä teknologioita käyttävillä ihmisillä on sekä laiskasta evaluoinnista yleisesti, että kysymyksen 2 tarkastelussa esiin tulleista ohjelmointikielistä ja apukirjastoista.

Menetelmänä kaikkiin kysymyksiin vastaamisessa on narratiivinen kirjallisuuskatsaus. Tavoitteena on kartoittaa, millaista tietoa aiheesta tällä hetkellä on, ja tuoda sitä yhteen helppotajuisen narratiivin muotoon. Tällä tavoin luotu katsaus soveltuu hyvin materiaaliksi opetustyössä \citep[s. 312]{baumeister1997writing}. Tämä sopii työni tavoitteisiin, sillä tavoitteena on nostaa tietämystä laiskasta evaluoinnista ja helpottaa siitä keskustelemista.

Kysymyksiin 1 ja 2 vastatakseni materiaalina käytän ensisijaisesti akateemisia artikkeleita, joita aiheesta on kirjoitettu paljon. Hain artikkeleita ensisijaisesti Scopus-tietokannasta, jossa käytin seuraavaa hakulauseketta:

\begin{listing}[H]
  \caption{Hakulauseke Scopus-tietokannasta}
  \bigskip
  \begin{minted}[linenos=false, xleftmargin=0pt]{python}
("lazy evaluation" OR "non-strict" OR "call-by-need" OR "call by need")
AND ("functional")
  \end{minted}
\end{listing}

Hakulausekkeessa käytetyt hakusanat esitellään luvussa \ref{kasitteisto}. Hakulauseke tuotti otsikosta, tiivistelmästä ja avainsanoista haettaessa 326 tulosta. Tämä oli sopiva määrä tuloksia tutkimuskysymysten kannalta relevanteimpien artikkeleiden valikointia ajatellen.

Huomattava osa tuloksista liittyi laiskan evaluoinnin matemaattisiin todistuksiin, jotka ovat olennaisia ohjelmointikielten kääntäjien suunnittelun kannalta, mutta eivät tämän työn rajauksen kannalta. Kuitenkin etenkin Haskell-ohjelmointikielen parissa työskentelevien ihmisten kirjoittamat artikkelit, joita tuloksissa oli myös paljon, osoittautuivat hyödyllisiksi.

Scopus-tietokannan lisäksi seuloin artikkeleita samalla hakutermillä myös Google Scholarista. Lisäksi Stack Overflow -kysymyspalvelun vastausten ja Haskellin oman wiki-alustan kautta löytyi useita linkkejä relevantteihin artikkeleihin. Täydensin akateemisista artikkeleista saatua tietoa myös blogikirjoituksilla. Niitä hyödynnän sellaisten uusien kehityskulkujen esittelyssä, joiden esiintyminen akateemisissa artikkeleissa on hyvin vähäistä. Käsitteiden määrittelyssä käytin myös alan yliopistotasoisia oppikirjoja lähteinä.

\begin{sloppypar}
Kysymykseen 3 vastatakseni käytän yhdistelmää eri tiedonhakutapoja. Hyödynnän (a) kirjallisuuskatsauksen tiedonhaussa vastaan tulleita mielipidelatautuneita artikkeleita, (b) blogikirjoituksia ja (c) mielipiteitä, joita olen kysynyt työkavereiltani Slack-kommunikaatioalustan avulla. Blogikirjoituksissa painotan kirjoittajia, joilla on myös akateemista näyttöä aihepiiristä. Mielipidettä kysyessä pyydän vastaajia kertomaan, millaisia vahvuuksia tai heikkouksia he ovat kohdanneet, jos he ovat käyttäneet laiskaa evaluointia ohjelmointikielissä tai niiden apukirjastossa.
\end{sloppypar}

Kirjallisuuskatsauksen ja mielipideselvityksen käsittelyjärjestystä tässä työssä esittelee seuraava osio, työn rakenne.
