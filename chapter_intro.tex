% DONE
% --------------------------------------------------------------------
% Cheatsheet:
% \verb!lazy val!   -- esim. lyhyisiin koodinpätkiin
% \begin{verbatim}  -- esim. pitkiin koodinpätkiin
% `` tekstiä ''     -- lainausmerkit
% \textbf{huom!}    -- boldaus
% \begin{quotation}
% \noindent \it     -- quoten lisäys
% \ldots            -- kolme pistettä
% footnote{juh}     -- alaviite
% \citet, \citep
% \citet[s.234]{}   -- viitteet, http://merkel.zoneo.net/Latex/natbib.php
% \begin{sloppypar} -- ahdas teksti
% \clearpage        -- onkkelmien korjaamiseen lukujen kanssa
% --                -- yhdysviiva
% \enumerate
% \itemize          -- listat
% \begin[htb]{figure}
% \begin[htb]{table} - Kuva ja taulukko, (h)ere, (t)op, (b)ottom
% \begin{equation}  -- kaava
% \label{eq:kaava1} -- laabeli
%!TEX root = main.tex

% DONE!}}
\section{Johdanto}

Laiska evaluointi on 1970-luvulta juontuva ohjelmointikielten suoritusta ohjaava periaate. Sen taustalla on ajatus siitä, että ohjelmakoodia suorittaessa kannattaisi välttää turhaa tai liian aikaista lausekkeiden evaluointia. Vastaavasti evaluointi kannattaa tehdä vasta, jos lausekkeen arvoa tarvitaan. Näin vältytään sekä tekemästä turhaa laskentaa että evaluoimasta sellaisia lausekkeita, jotka voisivat tuottaa ajonaikaisia ongelmia.

Periaatetta on luontevaa havainnollistaa esimerkillä. Kuvitellaan, että haluat laatia ohjelman jollain laiskan evaluoinnin periaatteita noudattavalla ohjelmointikielellä. Tämä ohjelma käsittelee listaa, joka koostuu aritmeettisista laskutoimituksista, ja tulostaa listan alkioita käyttäjälle. Ohjelmakoodi voisi näyttää seuraavalta:


\begin{listing}[H]
  \caption{Pseudokielinen esimerkki listaa käsittelevästä ohjelmasta}
  \bigskip
  \begin{minted}{python}
list = [2*3, 5^10, 100/2]
print(list[0])
print(list[2])
  \end{minted}
\end{listing}

Laiskan evaluoinnin periaatteita noudattaen ohjelman suoritus voisi tapahtua seuraavasti:

\begin{itemize}
    \item{Kun lista luodaan komennolla \verb!list = [2*5, 5^10, 100/2]!, mitään laskutoimitusta ei vielä evaluoida, koska se olisi liian aikaista.}
    \item{Kun listan alkioita tulostetaan, niiden arvot (25 ja 50) lasketaan vasta juuri ennen tulostamista.}
    \item{Koska lausekkeen \verb!5^10! arvoa ei tulosteta ohjelman suorituksen aikana, sitä ei koskaan evaluoida. Näin ylimääräiseltä työltä säästytään.}
\end{itemize}

Laiska evaluointi on periaatteena monille ohjelmoijille vieras, koska suosituista yleiskäyttöisistä ohjelmointikielistä ainoastaan Haskell käyttää sitä suoritusmallinaan. Merkittävästi yleisempi on ahneeksi evaluoinniksi kutsuttu periaate, jossa ohjelmakoodin suoritus etenee ohjelmakoodin kuvaamassa järjestyksessä, eikä laskentaa viivästetä myöhempään ajanhetkeen.

Monissa moderneissa ohjelmointikielissä on kuitenkin mahdollista käyttää laiskaa evaluointia joissain kielen ominaisuuksissa. Näistä esimerkkejä ovat Clojuren laiskat listarakenteet, Scalan laiskat instanssimuuttujat ja C++:n laiskat futuurit. Lisäksi joissain ohjelmointikielissä on suosittuja apukirjastoja, joilla laiskaa evaluointi voi simuloida. Tällainen on esimerkiksi JavaScript-yhteisössä suosittu Immutable.js, joka tarjoaa laiskasti evaluoituja listarakenteita.

Ohjelmointiyhteisön sisällä ymmärrys laiskasta evaluoinnista on vajavaista. Sen käytön hyödyistä ja haitoista on sirpaleista ja ristiriitaista tietoa. Lisäksi aihepiiristä keskusteleminen on hankalaa, koska laiskan evaluoinnin määritelmä ei ole yksiselitteinen. Näihin ongelmiin työni pyrkii vastaamaan seuraavien tutkimuskysymysten avulla:

\begin{enumerate}
  \item{Millaista käsitteistöä laiskan evaluoinnin yhteydessä käytetään?}
  \item{Mitkä ovat laiskan evaluoinnin sovellutuksia historiallisesti ja nykypäivänä?}
  \item{Millaisia etuja ja heikkouksia laiskaan evaluointiin liittyy?}
\end{enumerate}

Seuraavaksi käyn läpi tutkimuksessa käytetyt tutkimusmenetelmät joka tutkimuskysymyksen osalta. Sen jälkeen vastaan tutkimuskysymykseen 1 määrittelemällä keskeisimmät käsitteet luvussa \ref{kasitteisto}. Tutkimuskysymykseen 2 vastausta pohjustan luvussa \ref{historia}, joka käsittelee laiskan evaluoinnin historiaa. Se antaa lukijalle valmiudet ymmärtää laiskan evaluoinnin nykyisiä sovellutuksia, jotka ovat luvun \ref{sovellutuksia} aiheena. Luvussa \ref{subjektiiviset-edut} vastaan tutkimuskysymykseen 3. Luvussa \ref{yhteenveto} vedän työtä yhteen sekä käsittelen tulevaisuudennäkymiä ja mahdollisia jatkotutkimuksen aiheita.
