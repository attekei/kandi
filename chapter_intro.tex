% --------------------------------------------------------------------
% Cheatsheet:
% \verb!lazy val!   -- esim. lyhyisiin koodinpätkiin
% \begin{verbatim}  -- esim. pitkiin koodinpätkiin
% `` tekstiä ''     -- lainausmerkit
% \textbf{huom!}    -- boldaus
% \begin{quotation}
% \noindent \it     -- quoten lisäys
% \ldots            -- kolme pistettä
% footnote{juh}     -- alaviite
% \citet, \citep
% \citet[s.234]{}   -- viitteet, http://merkel.zoneo.net/Latex/natbib.php
% \begin{sloppypar} -- ahdas teksti
% \clearpage        -- onkkelmien korjaamiseen lukujen kanssa
% --                -- yhdysviiva
% \enumerate
% \itemize          -- listat
% \begin[htb]{figure}
% \begin[htb]{table} - Kuva ja taulukko, (h)ere, (t)op, (b)ottom
% \begin{equation}  -- kaava
% \label{eq:kaava1} -- laabeli
%!TEX root = main.tex

\section{Johdanto}

Laiska evaluointi on 1970-luvulta juontuva ohjelmointikielten suoritusta ohjaava periaate. Sen taustalla on ajatus siitä, että ohjelmakoodia suorittaessa kannattaisi välttää turhaa tai liian aikaista laskentaa. Vastaavasti laskenta kannattaa tehdä vasta, kun laskennan tulosta tarvitaan.

Periaatetta on luontevaa havainnollistaa esimerkillä. Kuvitellaan, että haluat laatia ohjelman jollain laiskan evaluoinnin periaatteita noudattavalla ohjelmointikielellä. Tämä ohjelma käsittelee listaa, joka koostuu aritmeettisista laskutoimituksista, ja tulostaa listan alkioita käyttäjälle. Ohjelmakoodi voisi näyttää seuraavalta:


\begin{listing}[H]
  \caption{Pseudokielinen esimerkki listaa käsittelevästä ohjelmasta}
  \bigskip
  \begin{minted}{python}
list = [2*3, 5^10, 100/2]
print(list[0])
print(list[2])
  \end{minted}
\end{listing}

Laiskan evaluoinnin periaatteita noudattaen ohjelman suoritus voisi tapahtua seuraavasti:

\begin{itemize}
    \item{Kun lista luodaan komennolla \verb!list = [2*5, 5^10, 100/2]!, mitään laskutoimitusta ei vielä lasketa, koska se olisi liian aikaista.}
    \item{Kun listan alkioita tulostetaan, niiden arvot (25 ja 50) lasketaan vasta juuri ennen tulostamista.}
    \item{Koska laskutoimituksen \verb!5^10! arvoa ei tulosteta ohjelman suorituksen aikana, sitä ei koskaan lasketa. Näin ylimääräiseltä laskennalta säästytään.}
\end{itemize}

Laiska evaluointi on periaatteena monille ohjelmoijille vieras, koska suosituista yleiskäyttöisistä ohjelmointikielistä ainoastaan Haskell käyttää sitä suoritusmallinaan. Merkittävästi yleisempi on ahneeksi evaluoinniksi kutsuttu periaate, jossa ohjelmakoodin suoritus etenee ohjelmakoodin kuvaamassa järjestyksessä, eikä laskentaa viivästetä myöhempään ajanhetkeen.

Monissa moderneissa ohjelmointikielissä on kuitenkin mahdollista käyttää laiskaa evaluointia joissain kielen ominaisuuksissa. Näistä esimerkkejä ovat Clojuren laiskat listarakenteet, Scalan laiskat instanssimuuttujat ja C++:n laiskat futuurit. Lisäksi joissain ohjelmointikielissä on suosittuja apukirjastoja, joilla laiskaa evaluointi voi simuloida. Tällainen on esimerkiksi JavaScript-yhteisössä suosittu Immutable.js, joka tarjoaa laiskasti evaluoituja listarakenteita.

Ohjelmointiyhteisön sisällä ymmärrys laiskasta evaluoinnista on vajavaista. Sen käytön hyödyistä ja haitoista on sirpaleista ja ristiriitaista tietoa. Lisäksi aihepiiristä keskusteleminen on hankalaa, koska laiskan evaluoinnin määritelmä ei ole yksiselitteinen. Näihin ongelmiin työni pyrkii vastaamaan.

\subsection{Tavoitteet ja tutkimuskysymykset}

Työni keskeisimpänä tavoitteena on koota laiskaa evaluointia käsittelevää kirjallisuutta yhteen siten, että aihetta on helpompi opettaa esimerkiksi yliopistoympäristössä ja siitä on mahdollista käydä korkeatasoista keskustelua yhteisellä, yksiselitteisellä terminologialla. Päädyin näiden tavoitteiden kautta seuraaviin tutkimuskysymyksiin.

\begin{enumerate}
  \item{Millaista käsitteistöä laiskan evaluoinnin yhteydessä käytetään?}
  \item{Mikä on laiskan evaluoinnin merkitys tietojenkäsittelytieteen historiassa ja nykypäivänä?}
  \item{Millaisia etuja ja heikkouksia laiskaan evaluointiin liittyy?}
\end{enumerate}

Kysymykseen 1 vastaan hakemalla yleisimpiä käsitteitä, joita laiskaa evaluointia käsittelevässä kirjallisuudessa esiintyy, hakemalla näille yleisimmät määritelmät, ja vertailemalla käsitteiden vaihtoehtoisia määritelmiä. Jäsentelen myös käsitteiden suhteita toisiinsa.

Kysymyksessä 2 laiskan evaluoinnin merkityksen tutkiminen tarkoittaa, että tarkastelen (a) laiskan evaluoinnin varhaista historiaa pohjustuksenomaisesti, (b) laiskan evaluoinnin leviämistä moderneihin ohjelmointikieliin, (c) laiskan evaluoinnin kautta kehittyneitä sovellutuksia ohjelmointikielissä ja ja (d) laiskaa evaluointia käyttävien teknologioiden hyödyntämistä tutkimuksessa ja työelämässä.

Kysymyksessä 3 selvitän sitä, millaisia subjektiivisia mielipiteitä laiskaa evaluointia hyödyntäviä teknologioita käyttävillä ihmisillä on sekä laiskasta evaluoinnista yleisesti, että kysymyksen 2 tarkastelussa esiin tulleista ohjelmointikielistä ja apukirjastoista.

\subsection{Työn rakenne}

[ to be rewritten. miten tästä saisi selkeän ja hyödyllisen lukijalle? ]

Ensiksi käyn läpi tutkimuksessa käytetyt tutkimusmenetelmät joka tutkimuskysymyksen osalta luvussa \ref{metodologia}. Sitten vastaan tutkimuskysymykseen 1 luvussa \ref{kasitteisto} määrittelemällä keskeisimmät käsitteet. Tutkimuskysymykseen 2 vastausta pohjustan luvussa \ref{historia}, joka käsittelee laiskan evaluoinnin historiaa ja keskeisempiä konsepteja. Se antaa lukijalle valmiudet ymmärtää laiskan evaluoinnin nykyisiä sovellutuksia, jotka ovat luvun \ref{sovellutuksia} aiheena. Luvussa \ref{subjektiiviset-edut} käsittelen subjektiivisesti koettuja etuja ja haittoja, joita laiskaan evaluointiin yleensä ja laiskan evaluoinnin sovellutuksiin liittyy. Luvussa \ref{yhteenveto} vedän työtä yhteen ja tarkastelen työn rajausta.
