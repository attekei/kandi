% ------------------------------------------------------------------
% Cheatsheet:
% \verb!lazy val!   -- esim. lyhyisiin koodinpätkiin
% \begin{verbatim}  -- esim. pitkiin koodinpätkiin
% `` tekstiä ''     -- lainausmerkit
% \textbf{huom!}    -- boldaus
% \begin{quotation}
% \noindent \it     -- quoten lisäys
% \ldots            -- kolme pistettä
% footnote{juh}     -- alaviite
% \citet, \citep
% \citet[s.234]{}   -- viitteet, http://merkel.zoneo.net/Latex/natbib.php
% \begin{sloppypar} -- ahdas teksti
% \clearpage        -- onkkelmien korjaamiseen lukujen kanssa
% --                -- yhdysviiva
% \enumerate
% \itemize          -- listat
% \begin[htb]{figure}
% \begin[htb]{table} - Kuva ja taulukko, (h)ere, (t)op, (b)ottom
% \begin{equation}  -- kaava
% \label{eq:kaava1} -- laabeli
%!TEX root = main.tex
\section{Käsitteiden määrittely}\label{kasitteisto}

Käyn tässä osiossa läpi käsitteistöä, joka on olennaista laiskan evaluoinnin aihepiirin ymmärtämiselle. Ensiksi käsittelen funktionaalista ohjelmointia, ohjelmointikielen kategoriaa, jossa laiskaa evaluointia erityisesti esiintyy. Sen jälkeen käyn läpi evaluointisemantiikkoja, jotka määrittelevät ohjelmointikielten funktioden ja funktiokutsujen noudattamia sääntöjä.

% Monille näistä käsitteistä on hankalaa hakea historiaa, ja historian mukaan jäsentely tuntuu keinotekoiselta, joten pidetään historiaosia erillisenä.
%	* Mageeta tässä osiossa on se, että jos tämä paisuu liian raskaaksi, niin käsitemäärittelyjä pystyy melko joustavasti ripottelemaan muihin osioihin sitä mukaa kun käsitteitä tulee vastaan
%	- [ ] Kirjoitetaan intro tähän osioon vasta sitten, kun käsitteet on määritelty
%		* Jotain tällaista: Eri kirjoittajat käyttäjät eri käsitteitä eri tavoin, ja sekaannuksia sekä käsitteiden sävyeroja on syytä tuoda eksplisiittisesti esiin

\subsection{Funktionaalinen ohjelmointi}
Funktionaalinen ohjelmointi on tapa rakentaa tietokoneohjelmia, joka on saanut inspiraationsa lamdakalkyylistä, matemaattisen logiikan mallista, jonka Alonzo Church laati 1930-luvulla \citep{church1932set}.

\citet{scott2009programming} sanoo, että tiukasti määriteltynä funktionaalisella tyylillä rakennettu ohjelma määrittelee ohjelman ulostulot niiden sisääntulojen funktiona. Funktiot ovat funktioita matemaattisessa mielessä, eli niillä ei ole sisäistä tilaa. Funktiot voivat kutsua toisiaan, ja siten ohjelmissa useat pienemmät funktiot yhdessä muodostavat ohjelman pääfunktion, joka muuntaa sisääntulot ulostuloiksi.

Funktionaaliseen ohjelmointiin liittyy läheisesti seuraavat käsitteet:
\begin{itemize}
	\item \textit{Deklaratiivinen ohjemointi} on funktionaalisen ohjelmoinnin yläkategoria, jossa keskiössä on sen kuvaaminen, mitä tietokoneen halutaan tekevän. Deklaratiivisille ohjelmointikielille on ominaista korkea abstraktiotaso ja se, että ohjelmoijan on luontevaa muotoilla ongelmansa kielen tarjoamilla abstraktioilla. Deklaratiivisen ohjelmoinnin vastakohta on imperatiivinen ohjelmointi, jossa puolestaan keskitytään siihen, miten tietokoneen kuuluisi suorittaa halutut tehtävät.

	\item \textit{Deterministisyys} tarkoittaa sitä, että tietyillä sisääntuloilla ohjelman ulostulo on aina sama riippumatta ajanhetkestä tai muista tekijöistä.

	\item \textit{Viittausten läpinäkyvyys} (eng. \textit{referential transparency}) tarkoittaa sitä, että ohjelmointikielen lausekkeen korvaaminen sen arvolla ei muuta ohjelman ulostuloa. Viittaukseltaan läpinäkyvät lausekkeet ovat deterministisiä.

	\item \textit{Sivuvaikutus} (eng. \textit{side effect}) tarkoittaa sitä, että ohjelmointikielen yhden aliohjelman kutsuminen vaikuttaa ohjelman muiden aliohjelmien palauttamiin arvoihin ohjelman myöhemmässä suoritusvaiheessa. Jos ohjelmassa on sivuvaikutuksia, se ei ole deterministinen, ja siten funktionaalisissa ohjelmointikielissä on mahdollisimman vähän sivuvaikutuksia.

	\item \textit{Puhdas funktio} (eng. \textit{pure function}) tarkoittaa funktiota, joka on sivuvaikutukseton, viittauksiltaan läpinäkyvä ja deterministinen.

	\item \textit{Puhtaasti funktionaalinen ohjelmointi} (eng. \textit{pure function}) tarkoittaa ohjelmointityyliä, jossa käytetään ainoastaan puhtaita funktioita.
\end{itemize}

Käsitteistä deterministisyys, viittausten läpinäkyvyys ja sivaikutuksettomuus ovat merkitykseltään osin päällekkäisiä, ja niitä kaikkia käytetään yleisesti funktionaalisen ohjelmoinnin luonteen kuvaamiseen.

Funktionaalinen ohjelmointityyli voi yksinkertaisimmillaan tarkoittaa sivuvaikutusten välttämistä, ja monet ohjelmointikielet tarjoavat tätä helpottavia työkaluja. Tällaisia kieliä ovat esimerkiksi Scala, Clojure ja Lisp. Jotkin kielet, kuten Haskell ja Miranda rakentuvat pitkälti puhtaasti funktionaalisen ohjelmointityylin ympärille, 

Useat ohjelmointikielet tarjoavat ainakin rajatun tuen funktionaaliselle ohjelmointityylille. 
% Usein ohjelmointikielet sisältävät funktionaalisen ohjelmoinnin piirteitä, mutta eivät tue puhtaasti funktionaalista ohjelmointityyliä. Tällaisia kieliä ovat esimerkiksi Scala, Clojure ja Lisp. Ohjelmointiyhteisössä näitä kuitenkin välillä kutsutaan hieman hämäävästi ``funktionaalisiksi kieliksi''.

\subsection{Evaluointisemantiikat}

Evaluointisemantiikka määrittelee säännöt ohjelmointikielen funktiokutsujen eri vaiheiden evaluoinnille. Käytännössä se vastaa yhteen tai useampaan seuraavista kysymyksistä:

\begin{itemize}
    \item Milloin funktiolle funktiokutsussa annetut parametrilausekkeet evaluoidaan?
    \item Millä tavoin evaluoidut parametrilausekkeiden arvot välitetään funktion ohjelmakoodille?
\end{itemize}

Kuvassa 1 on eriteltynä työni kannalta merkittävimmät evaluointisemantiikat. Kuvasta myös näkyy kuinka evaluointisemantiikat voi hahmottaa puuhierarkiana, jossa kukin hierarkian taso vastaa johonkin kysymykseen evaluoinnin noudattamista säännöistä.

Seuraavissa alaluvuissa kuvaan kunkin evaluointisemantiikan tyypillisimmän määritelmän ja kerron, mitä muita merkityksiä kyseiselle käsitteelle välillä kirjallisuudessa liitetään.


\begin{figure}[h]
  \begin{center}
	\footnotesize
	\begin{forest}
	for tree={
	  draw,
	  anchor=north,
	  align=center,
	  child anchor=north
	},
	[{Evaluointisemantiikat}, align=center, name=SS,s=0.3cm,l sep=0.7cm
	  [Tiukka evaluointi, name=PDC,l sep=0.7cm
		  [{Applikatiivinen evaluointi}, name=MS,l sep=0.7cm, tikz={\node [draw,label={[gray]below:{\small Ahne evaluointi}},dashed,gray,fit=()(!1)(!l)] {};}
		    [{Call-by-value}, name=MOODI]
		  ]
	  ]
	  [Ei-tiukka evaluointi,l sep=0.7cm
		[Normaalijärjestyksessä evaluointi,l sep=0.7cm, tikz={\node [draw,label={[gray]below:{\small Laiska evaluointi}},dashed,gray,fit=()(!1)(!l)] {};}
			[Call-by-name]
			[Call-by-need\\\scriptsize(call-by-name\\\scriptsize memoisoinnilla)]
		  ]
	  ]
	]
	\node[anchor=west,align=left] 
	  at ([xshift=-4.5cm]MOODI.west) {Miten parametrit\\välitetään koodille?};
	\node[anchor=west,align=left] 
	  at ([xshift=-4.5cm]MOODI.west|-MS) {Milloin parametrit\\evaluoidaan?};
	\node[anchor=west,align=left] 
	  at ([xshift=-4.5cm]MOODI.west|-PDC) {Mitä jos ei arvoa\\ alilausekkeella?};
	\path (current bounding box.west)|-coordinate(l)(MS.base);
	\end{forest}
\normalsize
	\caption{\footnotesize \textbf{Evaluointisemantiikkojen keskinäisiä suhteita puuhierarkialla havainnollistettuna.} Tiukka ja ei-tiukka semantiikka eivät ole varsinaisia evaluointisemantiikkoja, mutta ne on sisällytetty kuvaan, koska etenkin Haskell-yhteisössä niistä käytetään samassa yhteydessä evaluointisemantiikkojen kanssa. Ahne ja laiska evaluointi ovat epätarkasti määriteltyjä yleiskäsitteitä, joten niiden alle kuuluu useampi evaluointisemantiikka.}
    \label{figure:evaluation_semantics}
  \end{center}
\end{figure}

% - [ ] Jonkinlainen ryhmittelevä visualisointi / taulukko kärkeen jossa kaikki semantiikat (seuraavat bullet pointit) olisi esillä ja näkyisi miten ne suhtautuvat toisiinsa

\subsubsection{Tiukka ja ei-tiukka semantiikka}

Tiukka ja ei-tiukka semantiikka ovat etenkin Haskell-yhteisön käyttämiä termejä, jotka eivät suoranaisesti ota kantaa funktiokutsun eri vaiheiden evaluointiin. Ne ovat kuitenkin merkitykseltään läheisiä varsinaisille evaluointisemantiikoille, joten ne on luontevaa esitellä muiden evaluoiuntisemantiikkojen yhteydessä.

HaskellWikin (2017) mukaan \textit{tiukassa semantiikka} (eng. \textit{strict semantics}) tarkoittaa sitä, että ohjelmointikielen lausekkeella ei voi olla arvoa, jos millä tahansa sen alilausekkeella ei ole arvoa. \textit{Ei-tiukassa semantiikassa} (eng. \textit{non-strict semantics}) puolestaan lausekkeilla voi olla arvo, vaikka alilausekkeilla, joista nämä lausekkeet koostuvat, ei olisi arvoa.

Tilanne, jossa ohjelmointikielen lauseke ei saa arvoa, voi tarkoittaa käytännössä esimerkiksi ikuiseen silmukkaan jäämistä (kuten johdantoluvun \verb!noreturn! -funktiota kutsuessa) tai virheilmoitusta, joka lopettaa ohjelman suorituksen.

\subsubsection{Applikatiivinen evaluointi ja normaalijärjestyksessä evaluointi}

\citet{scott2009programming} mukaan \textit{applikatiivinen evaluonti} (eng. \textit{applicative evaluation}) tarkoittaa funktion parametrienlausekkeiden evaluointia ennen funktion ohjelmakoodin suorituksen aloittamista, ja ohjelmakoodille välitetään evaluoidut arvot. Listauksessa \ref{codepythonapplicative} on applikatiivisesta evaluoinnista Python-ohjelmointikielellä luotu esimerkki.

\begin{listing}[H]
  \caption{Esimerkki applikatiivisesta evaluoinnista Pythonilla}
  \label{codepythonapplicative}
  \bigskip
  \begin{minted}{python}
# Funktio, jonka suoritus jatkuu ikuisesti ja joka ei koskaan palauta arvoa
def noreturn(x):
  while True:
    x = -x
  return x

# Luodaan lista, jonka ainoan alkion muodostava lauseke ei koskaan palauta arvoa
# Lauseke evaluoidaan välittömästi, joten ohjelman suoritus ei etene
list = [noreturn(1)]
print "Tämä ei koskaan tulostu"
  \end{minted}
\end{listing}

\textit{Normaalijärjestyksessä evaluointi} (eng. \textit{normal-order evaluation}) puolestaan tarkoittaa, että parametrilausekkeet evaluoidaan vasta sitten, kun niitä oikeasti tarvitaan. Evaluoimattomat parametrilausekkeet välitetään ohjelmakoodille, ja vasta sitten kun ohjelmointikoodissa käytetään parametrien arvoa, lausekkeet evaluoidaan. Listaus \ref{codehaskellnormalorder} demonstroi normaalijärjestyksessä evaluointia Haskell-ohjelmointikielessä.

\begin{listing}[H]
  \caption{Esimerkki normaalijärjestyksessä evaluoinnista Haskellilla}
  \label{codehaskellnormalorder}
  \bigskip
  \begin{minted}{haskell}
-- Funktio, jonka suoritus jatkuu ikuisesti ja joka ei koskaan palauta arvoa
noreturn :: Integer -> Integer
noreturn x = negate (noreturn x)

-- Luodaan lista, jonka ainoan alkion muodostava lauseke ei koskaan palauta arvoa
-- Lauseketta ei vielä evaluoida listan määrittelyhetkellä
list = [noreturn 1]

-- Haskell-ohjelman suoritus tapahtuu `main` -päälausekkeessa
main = do
  -- Listan pituutta laskettaessa alkion arvoa ei tarvita,
  -- joten alkion muodostavaa lauseketta ei tarvitse evaluoida
  length list >>= print
  "Tämä tulostuu" >>= print
  -- Listan ensimmäisen arvon tulostaminen aiheuttaa lausekkeen evaluoinnin,
  -- joten ohjelman suoritus ei enää etene
  head list >>= print
  "Tämä ei koskaan tulostu" >>= print

  \end{minted}
\end{listing}

Applikatiivisessa evaluointi noudattaa tiukkaa semantiikkaa, sillä kaikilla funktioparametreilla (eli alilausekkeilla) on oltava arvo ennen kuin funktio itsessään evaluoidaan. Siten applikatiivisen evaluoinnin voi ajatella olevan tiukan semantiikan alakategoria evaluointisemantiikkojen hierarkiassa.

Vastaavasti normaalijärjestyksessä evaluointi, jossa evaluointia viivästetään, täytää ei-tiukan semantiikan kriteeristön, koska jokin parametri voi jäädä funktiokutsussa evaluoimatta, jos sitä ei tarvita funktion ohjelmakoodissa. Siinä tilanteessa ei ole merkitystä, olisiko tällä parametrilla ollut arvoa vai ei.


\subsubsection{Parametrimoodit}
% Call-By-Value, Call-By- Name, Call-By-Need
% (ehkä ylimääräisiä: Call-By-Sharing, Call-By-Reference)

% Vois tsekata vielä miten alkuperäisessä lähteessä tarkalleen näitä kuvataan
Parametrimoodeja yhdistää lähinnä niiden nimeämiskäytäntö, joka helpottaa niiden vertaamista keskenään. Yleisimmin käytettyjä parametrimoodeja ovat \citet{scott2009programming} mukaan

\begin{itemize}
	\item \textit{Call-by-value}, jossa funktiota kutsuttaessa parametrilauseke evaluoidaan ennen funktion ohjelmakoodia, ja arvo on käytettävissä parametria vastaavassa muuttujassa funktion ohjelmakoodissa. Jos parametrilausekkeena on ollut muuttuja, muuttujan arvo kopioidaan uuteen muuttujaan funktion ohjelmakoodia varten.
    \item \textit{Call-by-name}, jossa parametrilausekkeet sijoitetaan suoraan funktion ohjelmakoodiin niihin kohtiin, joissa argumentteihin käytetään. Parametrilauseke evaluoidaan uudestaan joka kerta, kun ohjelmakoodissa tarvitaan kyseisen argumentin arvoa.
    \item \textit{Call-by-need}, jossa parametrilauseke evaluoidaan vasta, kun funktion ohjelmakoodi tarvitsee sen arvoa ensimmäistä kertaa. Kun parametrilauseke on evaluoitu, se pidetään muistissa sitä varten, jos ohjelmakoodi tarvitsee argumenttia uudestaan. Usein tätä muistissa pitämisen periaatetta kutsutaan \textit{memoisoinniksi} (eng. \textit{memoization})
\end{itemize}

Call-by-value evaluoidaan applikoiden siinä missä call-by-name ja call-by-need evaluoidaan normaalijärjestyksessä. Siten parametrimoodit voi nähdä applikatiivisen evaluoinnin ja normaalijärjestyksessä evaluoinnin alakategorioina.

\subsubsection{Laiska ja ahne evaluointi}

Laiska evaluointi (eng. \textit{lazy evaluation}) esiintyy kirjallisuudessa tarkoittaen samaa kuin joko normaalijärjestyksessä evaluointi tai jokin sen alakategorioista. Myös \citet{scott2009programming} toteaa, että laiskaa evaluointia käytetään usein eräänlaisena yleiskäsitteenä useammille toisiaan muistuttaville evaluointisemantiikoille.

Ahne evaluointi (eng. \textit{eager evaluation}) määritellään usein laiskan evaluoinnin vastakohdaksi. Se voi kontekstista riippuen tarkoittaa samaa kuin esimerkiksi applikatiivinen evaluointi tai call-by-value -parametrimoodi.

Laiska ja ahne evaluointi ovat mielestäni molemmat käsitteinä ongelmallisia, koska ne ovat alttiita sekaannuksille sen suhteen, mitä niillä milloinkin tarkoitetaan. Siksi vältän niiden käyttöä tulevissa luvuissa, ja viittaan sen sijaan tarkemmin määriteltyihin evaluointistrategioihin.



% \begin{enumerate}
% 	\item Määritellään \verb!noreturn! -funktio, joka ei esimerkiksi ikuisen silmukan tai virhetilanteen takia koskaan palauta arvoa.
% 
% 	\item Muodostetaan lista, jonka alkioissa on ensin arvon saavia lausekkeita ja lopuksi kutsu \verb!noreturn! -funktioon. Yksinkertaisuuden nimissä käytetään arvon saavina lausekkeina numeroita.
% 
% 	\item Tarkistetaan ohjelmointikielen tarjoamalla komennolla, onko listassa jotain tiettyä arvoa, joka on 
% \end{enumerate}
% 
% Python-
% \begin{listing}[H]
%     \caption{Python, tiukan semantiikan säännöillä evaluoitu lauseke}
%   \bigskip
%   \begin{minted}[linenos=false, xleftmargin=0pt]{python}
% 2 in [2, 4, noreturn(5)]
%   \end{minted}
% \end{listing}
% 
% Haskellissa puolestaan lauseke noudattaa ei-tiukkaa semantiikkaa:

% Yksi tyypillinen esimerkki Esimerkkinä Pythonissa seuraava \verb!noreturn! -funktio jää aina ikuiseen silmukkaan eikä koskaan palauta arvoa:
%
% \begin{listing}[H]
%     \caption{Funktio joka ei koskaan saa arvoa Pythonissa}
%   \bigskip
%   \begin{minted}{python}
% def noreturn(x):
%   while True:
%     x = -x
%   return x
% \end{minted}
% \end{listing}
%
% Tämän funktion voi ilmaista denotatiivisen semantiikan avulla, joka on tapa formalisoida ohjelmointikielen käyttäytymistä matemaattisia ilmaisuja hyödyntäen. Denotatiivisen semantiikan kielellä tämän funktion käyttäytymisen voi ilmaista \textit{pohjatyypin} (eng. \textit{bottom type}) avulla, jota Haskell-yhteisössä merkataan tyypillisesti symbolilla \(\bot\). Pohjatyyppi tarkoittaa, että laskennan suoritus ei koskaan pääty onnistuneesti. \mbox{Pythonin} \verb!noreturn! -funktio merkitään
%
% \[
% noreturn\; x = \bot
% \]
%
% mikä tarkoittaa, että funktion suoritus ei pääty onnistuneesti millään muuttujan \(x\) arvolla. Samalla merkintä näyttää, että funktio noudattaa ei-tiukkaa semantiikkaa.
%
% Vastaavasti Haskell

% Kuvataan ero denotatiivisen semantiikan ja tarkemmin bottom typen / non-terminationin avulla; haasteena voi olla, että merkintätapa on lukijalle hiukkasen vieras. Onneksi verbaalisella selityksellä voi täydentää akateemista notaatiota.